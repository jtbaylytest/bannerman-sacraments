\PassOptionsToPackage{unicode=true}{hyperref} % options for packages loaded elsewhere
\PassOptionsToPackage{hyphens}{url}
%
\documentclass[]{book}
\usepackage{lmodern}
\usepackage{amssymb,amsmath}
\usepackage{ifxetex,ifluatex}
\usepackage{fixltx2e} % provides \textsubscript
\ifnum 0\ifxetex 1\fi\ifluatex 1\fi=0 % if pdftex
  \usepackage[T1]{fontenc}
  \usepackage[utf8]{inputenc}
  \usepackage{textcomp} % provides euro and other symbols
\else % if luatex or xelatex
  \usepackage{unicode-math}
  \defaultfontfeatures{Ligatures=TeX,Scale=MatchLowercase}
    \setmainfont[]{LiberationSerif}
    \setmonofont[Mapping=tex-ansi,Scale=0.7]{LiberationSerif}
\fi
% use upquote if available, for straight quotes in verbatim environments
\IfFileExists{upquote.sty}{\usepackage{upquote}}{}
% use microtype if available
\IfFileExists{microtype.sty}{%
\usepackage[]{microtype}
\UseMicrotypeSet[protrusion]{basicmath} % disable protrusion for tt fonts
}{}
\IfFileExists{parskip.sty}{%
\usepackage{parskip}
}{% else
\setlength{\parindent}{0pt}
\setlength{\parskip}{6pt plus 2pt minus 1pt}
}
\usepackage{hyperref}
\hypersetup{
            pdftitle={The Sacraments},
            pdfauthor={James Bannerman},
            pdfborder={0 0 0},
            breaklinks=true}
\urlstyle{same}  % don't use monospace font for urls
\usepackage{longtable,booktabs}
% Fix footnotes in tables (requires footnote package)
\IfFileExists{footnote.sty}{\usepackage{footnote}\makesavenoteenv{longtable}}{}
\usepackage{graphicx,grffile}
\makeatletter
\def\maxwidth{\ifdim\Gin@nat@width>\linewidth\linewidth\else\Gin@nat@width\fi}
\def\maxheight{\ifdim\Gin@nat@height>\textheight\textheight\else\Gin@nat@height\fi}
\makeatother
% Scale images if necessary, so that they will not overflow the page
% margins by default, and it is still possible to overwrite the defaults
% using explicit options in \includegraphics[width, height, ...]{}
\setkeys{Gin}{width=\maxwidth,height=\maxheight,keepaspectratio}
\setlength{\emergencystretch}{3em}  % prevent overfull lines
\providecommand{\tightlist}{%
  \setlength{\itemsep}{0pt}\setlength{\parskip}{0pt}}
\setcounter{secnumdepth}{5}
% Redefines (sub)paragraphs to behave more like sections
\ifx\paragraph\undefined\else
\let\oldparagraph\paragraph
\renewcommand{\paragraph}[1]{\oldparagraph{#1}\mbox{}}
\fi
\ifx\subparagraph\undefined\else
\let\oldsubparagraph\subparagraph
\renewcommand{\subparagraph}[1]{\oldsubparagraph{#1}\mbox{}}
\fi

% set default figure placement to htbp
\makeatletter
\def\fps@figure{htbp}
\makeatother

\usepackage{booktabs}
\usepackage{amsthm}
\makeatletter
\def\thm@space@setup{%
  \thm@preskip=8pt plus 2pt minus 4pt
  \thm@postskip=\thm@preskip
}
\makeatother
\usepackage[]{natbib}
\bibliographystyle{apalike}

\title{The Sacraments}
\author{James Bannerman}
\date{1868}

\begin{document}
\maketitle

{
\setcounter{tocdepth}{1}
\tableofcontents
}
\hypertarget{a-warhorn-historicals-book}{%
\chapter*{\texorpdfstring{\href{https://historicals.warhornmedia.com/}{A Warhorn Historicals Book}}{A Warhorn Historicals Book}}\label{a-warhorn-historicals-book}}
\addcontentsline{toc}{chapter}{\href{https://historicals.warhornmedia.com/}{A Warhorn Historicals Book}}

\includegraphics[width=0.5\textwidth,height=\textheight]{images/warhornlogo.png}

Republished by Warhorn Historicals:
Making historical books available for free online in high quality, readable formats.

We hope this book is a blessing to you. If it is, please \href{https://warhornmedia.com/give}{make a one-time or recurring contribution} right now, sponsor a book from our upcoming list, or volunteer your proofreading or technical skills to help produce more content. Contact \href{mailto:lucas@beggarsborn.com}{Lucas Weeks} to get involved.

God bless,

---The Warhorn Team

\hypertarget{cover}{%
\subsubsection*{Cover}\label{cover}}
\addcontentsline{toc}{subsubsection}{Cover}

\hypertarget{introduction}{%
\chapter*{Introduction}\label{introduction}}
\addcontentsline{toc}{chapter}{Introduction}

Nobody understands the Church like Bannerman, and this section on the Sacraments is mandatory reading today.

\hypertarget{the-sacraments-in-general}{%
\chapter{The Sacraments In General}\label{the-sacraments-in-general}}

FOR some time past we have been occupied with the subject of the ordinances of the Christian Church. We have discussed the questions connected with the public worship appointed in the Church, the special time set apart and sanctified for worship, and the ministry by means of which the worship of the Church is conducted. All these are outward ordinances which Christ has established in His Church, as parts of that external provision which He has made for the spiritual benefit and advancement of His people, and which He specially makes effectual to that end by the presence and power of His Spirit. All of these ordinances are in themselves, perhaps, and naturally adapted by their inherent character and influence to promote the edification of Christians; but above and beyond this natural or moral efficacy for that end, there is a spiritual blessing connected with them in consequence of the positive appointment of Christ, and the positive promise of His Spirit fulfilled in the right use of them. There may be a natural or moral efficacy in the ordinances of the Church considered in themselves, so that, apart from any other influence, they would, to a certain extent, be beneficial and advantageous in the case of those who used them. But in addition to this, there is a spiritual efficacy in the ordinances of the Church, distinct from the natural, and which is derived from the blessing of Christ and the working of His Spirit in them who by faith make use of them as He has appointed. What this spiritual and supernatural efficacy of outward ordinance exactly is,---what is the measure or amount of the inward benefit to the believer,---in what way and to what extent grace is connected with the external observance,---how beyond the sphere of this natural or moral influence the positive institutions of the Church have a blessing not natively their own,---these are questions which it is impossible for us distinctly to answer. The only wise and fitting reply to such questions is, that we have now reached the region of the supernatural, and that \emph{there} we have no data to guide us beyond what has been revealed. We know, from revelation, that there is a promise of grace annexed to outward ordinances when rightly used; we know that in the external observances Christ meets with His people to bless them and to do them good;---but beyond this we do not know. The character, the measure, the amount of the blessing promised,---how it stands connected with the outward ordinance, and what is the extent and efficacy of the supernatural grace over and above the natural efficacy of the ordinance,---of all this we know nothing, because we have been told nothing. We can distinctly understand, from the analogy of other cases, how the preaching of the Word, viewed as a system of human teaching of truth, and no more, may have a natural tendency to benefit the understanding and the heart. But we do not understand the supernatural efficacy which, over and above the natural, is imparted to it by the presence and the power of the Spirit in the ordinance.

In passing, as we do at this stage, from the non-sacramental to the sacramental ordinances appointed by Christ in His Church, it is of great importance to carry this general principle along with us. A supernatural grace is not peculiar to the Sacraments, although it may be found in them in larger measure than in other ordinances. It is common to all the ordinances which Christ has appointed in His Church. Whatever mystery there may be in the connection which by the promise of Christ has been established between the outward act and the inward blessing,---between the external observance rightly used and the internal grace divinely bestowed,---it is a mystery not belonging to Sacraments alone, but belonging to them in common with all Church ordinances. \emph{There is the supernatural element in them all}. There is that supernatural element connected in some manner with the outward act of the believer in the use of ordinances. There is a mystery in respect to any ordinance, not less than in respect of sacramental ordinances, which we cannot explain. It is, in short, the mystery of the Spirit of God, promised to dwell in the Church, and making every ordinance of the Church, whether sacramental or not, the channel for the conveyance of supernatural grace. If we would rid ourselves of this mystery, we can only do so by denying that the Spirit is present in ordinances at all. ``As the wind bloweth where it listeth, and thou hearest the sound thereof, but canst not tell whence it cometh, and whither it goeth,''---so is every ordinance, as well as each person, touched and sanctified of the Holy Ghost. There can be no natural explanation of the supernatural.\footnote{{[}Bannerman, \emph{Inspiration: The infallible Truth and Divine Authority of the Holy Scriptures}, Edin. 1865, pp.~217-228, 472 f.{]}}

What, then, is the character of those special ordinances instituted by Christ in His Church, which are usually denominated sacramental ordinances; and in what respect are they to be distinguished from the other ordinances of the Christian Church, not sacramental? In administering Sacraments, what is the peculiar nature or character of the Church's act; and in what manner does the administration differ from that of common ordinances?

The term \emph{Sacrament}, by which these peculiar ordinances are known, is not of scriptural, but of ecclesiastical origin; and there is some doubt as to the manner in which it came to be applied to these special solemnities of the Church, and to be restricted to the peculiar meaning in which it is now almost universally employed. In classical use, the word ``\emph{sacramentum}'' is almost always, if not invariably, employed to signify an oath,---more especially the military oath by which a soldier bound himself to obey the officer placed over him. And it has been conjectured that from its classical use it was transferred into the service of the Church, as significant of the obligation which the Christian comes under, in voluntarily participating in the Sacraments, to serve Christ as the Captain of his salvation,---these Sacraments being the characteristic badges or symbols by which the Christian is distinguished from other men. There is a second explanation, advocated by not a few, of the way in which the Latin term Sacrament came to be appropriated to its present ecclesiastical sense. It is the ordinary translation of the Greek word \emph{μυστηριον} among the ecclesiastical writers of the early ages, and more especially in the Vulgate and other old Latin translations of the Bible. The term Sacrament, according to this supposition, came to be employed to signify the ``mysteries'' of Christianity,---whether ``mystery'' is employed to denote a doctrine unknown until it was revealed, or a type or emblem bearing a hidden and secret meaning.\footnote{Turrettin, \emph{Opera}, loc. xix. qu. i. 1-6. Halley, \emph{The Sacraments}, Lond. 1844, pp.~7-14.} There is some reason to believe that both the Greek term \emph{μυστηριον} and the Latin translation of it---sacramentum---came at an early period to be applied by the primitive Christians to those special solemnities of their faith, which, although made up of outward and sensible signs or actions, bore in them a secret and spiritual meaning. In one or other of these ways, or perhaps in both, the term ``Sacrament'' soon came to be restricted in its meaning and application, by ecclesiastical practice, to those outward ordinances of Christianity which signify and seal its most precious and momentous truths. But as the term itself is of Church origin, and not found in Scripture, we must look not to it, but to the descriptions and intimations given in Scripture in regard to the ordinances themselves, for an explanation of their true nature and import.\footnote{{[}``The Apostle calleth the vocation of the Gentiles a mystery (Eph. iii. 4-6); the conjunction quhilk is begun here betwixt us and Christ is called a mystery (Eph. v. 32), and the Latin Interpreters call it a Sacrament; and, to be short, ye will not find in the Book of God a word mair frequent nor the word mystery. . . . Alwayis, the word Sacrament is very ambiguous in itself, and there raise about the ambiguity of this word many tragedeis quhilk are not yet ceased, nor will cease while the warld lasts; quher otherwise, gif they had keeped the Apostle's words, and called them, as the Apostle calls them, signs and seals, all this digladiatioun, strife, and contention appearandly had not fallen out. But quher men will be wiser than God, and give names to things beside God, upon the wit of man, quhilk is but mere folly, all this cummer falls out. . . . The ancient theologues took the word Sacrament in a fourfold manner. Sometimes they took it for the hail action, that is, for the hail ministrie of the elements. Sometimes they took it, not for the hail action, but for the outward things that are used in the action of Baptism and of the Supper; as they took it for the water and sprinkling of it, for the bread and wine, breaking, distributing, and eating thereof. Again, they took it, not for the hail outward things that are used in the action, but only for the material and earthly things,---the elements; as for bread and wine in the Supper, and water in Baptism. After this sort sayeth Augustine: `The wicked eats the body of our Lord concerning the Sacrament only;' that is, concerning the elements only. (Aug.~in \emph{Joann}. Tract xxvi. 18). Last of all, they took it not only for the elements, but for the things signified by the elements. And after this manner, Irenaeus saith, `that a Sacrament stands of twa things,---the ane earthly, the other heavenly.' (Adv. \emph{Haeres}. lib. iv. cap. 18.) The ancients, then, taking the word after thir sorts, na question all thir ways they took it rightly.''---Robert Bruce, \emph{Sermons on the Sacraments}, p.~6, Wodrow Soc. ed. Edin. 1843.{]}} In what respects, then, do the Scriptures represent the Sacraments of the Church as differing from its other ordinances which are not sacramental? What, according to Scripture, must we regard as the true nature and design of a Sacrament? To this general question we shall direct our attention in the first place, postponing for the present the special consideration of the Sacraments individually. And in endeavouring to ascertain the real nature and design of the Sacraments of the New Testament, we shall be enabled to understand at the same time, and by means of the same inquiry, in what respects they differ from other ordinances not sacramental.

\hypertarget{nature-and-efficacy-of-the-sacraments-of-the-new-testament-and-difference-between-them-and-non-sacramental-ordinances}{%
\section{Nature And Efficacy Of The Sacraments Of The New Testament, And Difference Between Them And Non-Sacramental Ordinances}\label{nature-and-efficacy-of-the-sacraments-of-the-new-testament-and-difference-between-them-and-non-sacramental-ordinances}}

\hypertarget{the-sacraments-of-the-new-testament-are-divine-institutions-appointed-by-christ.}{%
\subsection{The Sacraments of the New Testament are Divine institutions appointed by Christ.}\label{the-sacraments-of-the-new-testament-are-divine-institutions-appointed-by-christ.}}

It is the positive institution by Christ that sets these ordinances apart to the religious purpose for which they are intended, that makes them significant of spiritual things, and connects them with the virtue or blessing which they are made instrumental to impart. An express Divine appointment is necessary to constitute a Sacrament. In this respect they are similar to the other ordinances which form part of Church worship. Like them, they can claim Divine authority for their institution; and without this authority they would not be Sacraments at all. No observance not ordained by God can properly form any part of His service; far less can any observance not instituted by Him become a sign of His spiritual grace, or a pledge of a blessing which it depends upon His pleasure to give or to withhold. Hence, that any outward institution may answer to our idea of a Sacrament, it must be a positive appointment of God, and made both a sign and a pledge of spiritual blessings, in consequence of His promise and command. Without this, it would be a mere human ordinance, not only destitute of all real religious significance and efficacy, but profanely mimicking the form and character of a Divine ordinance in the Church. This is the first element that goes to make up a Sacrament, and which it has in common with all other ordinances, really forming a lawful or proper part of Divine worship,---namely, that it be of positive appointment by Christ.

\hypertarget{the-sacraments-of-the-new-testament-are-sensible-signs-of-spiritual-blessings-teaching-and-representing-by-outward-actions-gospel-truths.}{%
\subsection{The Sacraments of the New Testament are sensible signs of spiritual blessings, teaching and representing by outward actions Gospel truths.}\label{the-sacraments-of-the-new-testament-are-sensible-signs-of-spiritual-blessings-teaching-and-representing-by-outward-actions-gospel-truths.}}

The word or promise of God is an appeal to the understanding only; the Sacraments, embodying the same word or promise in outward and sensible signs, form a twofold appeal, \emph{first}, to the senses, and \emph{secondly}, to the understanding. There is Christ in the Word preached; and in the preaching of the Word, Christ is presented directly to the understanding and heart, and the truth addressed singly to the spiritual nature of man. But Christ is also in the Sacrament administered; and, in the administration of the Sacrament, over and above the same truth taught to the understanding and spiritual nature of man, there is the truth taught to the senses, and impressed by sensible signs upon them. There is a striking similarity between the method God has employed in the Sacraments of the New Testament to embody the Word and promises of Christ, and of a \emph{past} salvation, to the view of His people since His departure, and the method that He employed before Christ's coming to embody the Word and promises of a \emph{future} salvation. Under the Old Testament Church, there were, from the very first, two lines of promise and prediction,---both pointing forward to the coming of the Redeemer, running parallel with each other, and throwing mutual light upon each other's announcements. There was the line of promise embodied in verbal revelation, and there was the line of promise embodied in outward representation or type.

These two revelations ran parallel with each other since the first hour that a revelation was given to man in Paradise concerning the future coming of a Saviour. At that time there was a promise embodied in words, that ``the woman's seed should bruise the serpent's head, while His own heel was to be bruised;'' and side by side with that verbal announcement, there was the same promise embodied in type through means of the ordinance of sacrifice then appointed. There was Christ in the word of promise, and Christ in the sign of promise. When the promise was renewed to Noah, the second father of the human family, we have again the revelation by word, and the revelation by sensible sign; the covenant was repeated in another form, and the bow was set in the cloud as the outward representation of it. Once more: when Abraham was selected by God to be the depositary of a new development of the promise, we have again that promise embodied in words, and also in outward action; we have the special covenant with Abraham revealed in words, and revealed side by side with the word in the external sign of circumcision; and---to mention no further examples of a practice which must be familiar to every reader of the Old Testament---the whole of the Jewish economy was an exemplification of the two parallel lines that run through every economy of God,---the promise in word and the promise in sign revealed together, and throwing mutual light on each other. The typology of the Old Testament shows us God embodying His promises in signs; the revelation of the Old Testament shows us God embodying the same promises in words; and the Sacraments of the New Testament afford, under the Gospel economy, an exemplification of the same great principle.

The connection between the outward action in the Sacraments and the spiritual blessings to which they stand related is not a mere arbitrary one, arising from positive institution: there is a natural analogy or resemblance between the external signs and the things represented; so that, in the Sacraments of the New Testament, as in the types of the Old, our senses are made to minister to our spiritual advantage, and the outward action becomes the image of inward grace. In the Word, Christ is impressed on the understanding; in the Sacraments, Christ is impressed both on the understanding and the senses. They become teaching signs, fitted and designed to address to the believer the very same truths as are addressed to him in the Word; but having this peculiarity, that they speak at the same time and alike to the outward senses and to the inward thought. In this respect the Sacraments differ from other ordinances of the New Testament Church. Prayer and preaching and praise are ordinances that address themselves to the intellectual and spiritual nature of man alone. They are the expressions and utterances of his intellectual and spiritual being in holding intercourse with God; or they are the means fitted to speak to that nature, and that only, in impressing Divine truth upon men. But in those significant and teaching signs, which we call the Sacraments, Christ is embodied in the ordinance in such a manner as to appeal to the twofold being of man, as made up of body and soul, to minister both to the senses and the understanding; and to speak at once to the outward and inward nature of the believer. In addition to Christ in the Word, we have Christ also in the sign, taught as really in the latter way as in the former, and taught with the advantage of being submitted to the eye, and pictured to the outward senses. This, then, is one important difference between the sacramental ordinances of the New Testament Church and those which are not sacramental.

\hypertarget{the-sacraments-of-the-new-testament-are-federal-acts-affording-a-seal-or-confirmation-of-the-covenant-between-god-and-his-people.}{%
\subsection{The Sacraments of the New Testament are federal acts affording a seal or confirmation of the covenant between God and His people.}\label{the-sacraments-of-the-new-testament-are-federal-acts-affording-a-seal-or-confirmation-of-the-covenant-between-god-and-his-people.}}

This is the main and primary characteristic of sacramental ordinances. They constitute a formal testimony to an engagement entered into by two parties through means, not of words, but of speaking and significant actions,---these actions being the visible witnesses to the engagement, and the outward confirmations of its validity. In other words, they become, according to the expression of the apostle in his Epistle to the Romans, when speaking of one of the Sacraments of the Old Testament, visible ``\emph{seals}'' of the covenant, and of the blessings contained in it.\footnote{Rom. iv. 11.}

There are not a few examples to be found in the Old Testament Scriptures of covenants between man and man ratified by some outward monument, framed or chosen to attest and confirm the transaction. When Jacob parted from his father-in-law Laban, they made a covenant together, and raised a heap of stones and a pillar, to be a memorial of the transaction, and to serve as a witness on both sides to attest their fidelity to the terms of the covenant. ``This heap be a witness, and this pillar be a witness, that I will not pass over this heap to thee, and that thou shalt not pass over this heap and this pillar to me, for harm.''\footnote{Gen.~xxxi. 52.} The outward monument or memorial of the covenant entered into between Jacob and Laban was a witness of the engagement, serving to bind the obligation of it more strongly on both parties, and to ratify and confirm, in a formal and significant manner, its validity. And what we find in patriarchal times, we also find, in one shape or other, in every stage of society, some outward sign or significant action being made use of between men to confirm and attest their plighted faith. In addition to the spoken promise or oath, there has been---if not the stone of the times of Jacob---at least the formal signature and solemn deed, and seal attached to the deed, to remain after the verbal engagement, as the witness and ratification of the transaction. Such outward monuments or significant solemnities are intended for the satisfaction of both parties, and to give additional certainty and confirmation to the agreement. And the practice in this respect, which has obtained universally among men, we find to be made use of also by God. There are repeated examples in the Old Testament Scriptures of God ratifying His engagements or covenants with men by means of appropriate signs or solemnities, and making use of these solemnities for the very same purpose that a signed and sealed deed is employed for in the present day, when it attests or confirms a previous engagement, and gives additional security to both parties for the fulfilment of it. That in such a sense the rainbow in the cloud was employed by God, when it became the sign of His covenant with Noah, is very expressly stated by Himself: ``And the bow shall be in the cloud; and I will look upon it, that I may remember the everlasting covenant between God and every living creature of all flesh that is upon the earth. And God said unto Noah, This is the token of the covenant, which I have established between me and all flesh that is upon the earth.''\footnote{Gen.~ix. 16, 17.} In this point of view the bow was a seal, giving validity and additional security to the covenant then made, and serving as a standing witness for the truth of it. In a precisely similar manner, the rite of circumcision was appointed to Abraham for a voucher of the covenant between God and him. The terms of the institution of the rite would themselves lead us to this conclusion, even had they not been interpreted by the inspired commentary of the Apostle Paul in that sense. ``And, God said unto Abraham, Thou shalt keep my covenant therefore, thou, and thy seed after thee in their generations. This is my covenant, which ye shall keep, between me and you, and thy seed after thee. Every man-child among you shall be circumcised. And ye shall circumcise the flesh of your foreskin; and it shall be a \emph{token of the covenant} betwixt me and you.'' And in reference to this transaction, the Apostle Paul expressly says of Abraham: ``And he received the sign of circumcision, \emph{a seal of the righteousness of the faith} which he had yet being uncircumcised.''\footnote{Gen.~xvii. 9-11; Rom. iv. 11.} The outward act of circumcision, then, was a witness or a seal of the covenant transaction between God and the patriarch, and thus became a voucher to ratify and confirm the validity of it.

In exact accordance with the practice, universal in one shape or other among men, and expressly sanctioned by the example of God Himself in the Old Testament Church, we affirm that the Sacraments of the New Testament are parts of a federal transaction between the believer and Christ, and visible and outward attestations or vouchers of the covenant entered into between them. In addition to being signs to represent the blessings of the covenant of grace, they are also seals to vouch and ratify and confirm its validity. That the Sacraments of the Christian Church are thus seals of the covenant, appears to be very explicitly asserted, in so far at least as regards the Lord's Supper, in the words of the institution themselves: ``\emph{This} cup,'' said our Lord, ``\emph{is the new covenant in my blood, which is shed for you},''\footnote{Luke xxii. 20.}---language which seems undoubtedly intended to convey the idea that the element used in the Supper was to be the witness of the new covenant,---a visible seal or security to ratify and vouch for it. No doubt that covenant in itself is sufficiently secure without any such confirmation, resting as it does on the word of God. That word alone, and without any further guarantee, is enough. But in condescension to the weakness of our faith, and adapting Himself to the feelings and customs of men, God has done more than give a promise. He has also given a guarantee for the promise,---has vouchsafed to bestow an outward confirmation of His word in the shape of a visible sign, appealing to our senses, and witnessing to the certainty and truth of the covenant. In the case of the Sacraments, God has proceeded on the same principle as is announced by the Apostle Paul in reference to His oath: ``God, willing more abundantly to show unto the heirs of promise the immutability of His counsel, confirmed it by an oath; that by two immutable things, in which it was impossible for God to lie, we might have a strong consolation, who have fled for refuge to lay hold upon the hope set before us.''\footnote{Heb. vi. 17, 18.} The word of promise was itself enough to warrant and demand the belief of God's people. But more than enough was granted: He has not only said it, but also sworn it. By two immutable things---His word and His oath---is the faith of the believer confirmed. The oath is the guarantee for His word. And more than this still: In the visible seal of the Sacraments God would add another and a third witness,---that at the mouth, not of two, but of three witnesses, His covenant may be established. He has not only given us the guarantee of His word, and confirmed that word by an oath, but also added to both the seal of visible ordinances. There is the word preached to declare the truth of the covenant to the unbelieving heart. More than that,---there is the oath sworn to guarantee it. More than that still,---there is the sign administered in order to vouch for all. Christ in the word, unseen but heard, is ours, if we will receive that word with the hearing ear and the understanding heart. Over and above this, Christ, both seen and heard in the Sacrament, is ours, if we will see with the eye or hear with the ear.\footnote{{[}``What mister (need) is there that thir Sacraments and seals suld be annexed to the Word? Seeing we get na new thing in the Sacrament but the same thing quhilk we gat in the simple Word, quherefore is the Sacrament appointed to be hung to the Word? It is true certainly, we get na new thing in the Sacrament, nor we get na other thing in the Sacrament nor we gat in the Word; for \emph{quhat mair walde thou crave nor to get the Son of God}, gif thou get Him weil? Thy heart cannot wish nor imagine a greater gift nor to have the Son of God, quha is King of heaven and earth. And therefore I say, quhat new thing walde thou have? For gif thou get Him, thou gettest all things with Him. Quherefore, then, is the Sacrament appointed? Not to get thee a new thing. I say it is appointed to get thee that same thing \emph{better} nor thou gat it in the Word. The Sacrament is appointed that we may get a better grip of Christ nor we gat in the simple Word; that we may possess Christ in our hearts and minds mair fully and largely nor we did of before in the simple Word; that Christ might have a larger space to make residence in our narrow hearts nor we could have by the hearing of the simple Word. And to possess Christ mair fully it is a better thing; for suppose Christ be ae thing in Himself, yet the better grip thou have of Him thou art the surer of His promise.''---Bruce, \emph{Sermons on the Sacraments}, Wodrow Soc. ed. Edin. 1843, p.~28.{]}}

The Sacraments are the outward and sensible testimony and seal of the covenant, added to the word that declares it. This is the grand peculiarity of sacramental ordinances, separating them by a very marked line from ordinances not sacramental. They are federal acts,---seals and vouchers of the covenant between God and the believer. They presuppose and imply a covenant transaction between the man who partakes of them and God; and they are the attestations to and confirmations of that transaction, pledging God by a visible act to fulfil His share of the covenant, and engaging the individual by the same visible act to perform his part in it. Other ordinances, such as the preaching of the Word, presuppose and attest no such personal engagement or federal transaction between the individual and God. Christ in the Word is preached to all, and all are called upon to receive Him; but there is no personal act on the part of the hearer that singles him out as giving or receiving a voucher of his covenant with his Saviour. But when the same individual partakes of the Sacraments, his own personal deed is an act of covenanting with God; and Christ in the ordinance is made his individually, and he is made Christ's by the very action of partaking of the ordinance. He is singled out by his own voluntary act, if he rightly partakes of the ordinance, as giving a voucher for his engagement with Christ; and Christ Himself gives a voucher of His engagement to the individual; and the visible Sacrament is the seal to the personal and mutual engagement. In this respect, as not only signs but seals of the covenant of grace to the individual who in faith partakes of them, the Sacraments are very markedly distinguished from ordinances not sacramental.

\hypertarget{the-sacraments-of-the-new-testament-are-made-means-of-grace-to-the-individual-who-rightly-partakes-of-them.}{%
\subsection{The Sacraments of the New Testament are made means of grace to the individual who rightly partakes of them.}\label{the-sacraments-of-the-new-testament-are-made-means-of-grace-to-the-individual-who-rightly-partakes-of-them.}}

It is carefully to be noted that they presuppose or imply the possession of grace in the case of those who partake of them; but they are also made the means of adding to that grace. They are seals of a covenant already made between the soul and Christ,---attestations of a federal transaction before completed,---confirmations, visible and outward, of engagement between the sinner and his Saviour previously entered into on both sides. They presuppose the existence of grace, else they could not be called seals of it. Just as the signature and seal of some human covenant necessarily presuppose that the covenant exists before they can become vouchers for it, so the seal of God's covenant, affirmed by means of sacramental ordinances, presupposes the existence of that covenant as already subsisting between God and the rightful participator in the ordinance. But although grace exists in the soul before, the Sacraments are made to those who rightly receive them the means of increasing that grace, and communicating yet more of spiritual blessing. They serve to strengthen the faith of those who already believe, and add to the grace of those who previously possessed grace. They become effectual means of imparting saving blessings in addition to those enjoyed before.\footnote{{[}``The Church has always seen in the Sacraments,'' says Mr.~Liddon in his recent very valuable work on the Divinity of our Lord, ``not mere outward signs addressed to the taste or imagination, nor even signs, as Calvinism asserts, which are tokens of grace received independently of them, but signs which, through the power of the promise and Word of Christ, effect what they signify.'' For this very defective statement of the Calvinistic doctrine of the Sacraments the only authority Mr.~Liddon gives is a single secondhand quotation from Cartwright. He then proceeds to contrast with this supposed Calvinistic view the words of the 25th Article: ``The Sacraments are \emph{effectual} signs of grace and God's goodwill toward us, by which He doth work invisibly in us;'' and the definition of the Church Catechism: ``A Sacrament is an outward and visible sign of an inward and spiritual grace given unto us, ordained by Christ Himself as a \emph{means whereby} we receive the same, and a pledge to assure us thereof.'' \emph{Bampton Lectures}, 1866, p.~721. A very slight reference to the symbolical books or the leading theologians of some of the Calvinistic Churches would of course have shown that all these phrases have been constantly used by them with respect to the Sacraments. The isolated sentence from Cartwright adduced by Hooker, from whom Mr.~Liddon takes it, refers to a particular aspect of a particular Sacrament; it was never designed to be a full definition of the efficacy of these ordinances in a typical case. Moreover, the passage in question is just a translation of Calvin, \emph{Inst}. iv. xv. 22. It might as well, therefore, have been brought forward as expressive of \emph{his} whole doctrine on the subject. But Mr.~Liddon must surely be aware that Calvm constantly speaks of the Sacraments both of the Old and New Testaments as ``\emph{effectual} means of grace,'' ``efficacious instruments,'' ``signs in which God \emph{gives} what He holds out to us,'' etc. (``non modo salutaria exercitia, et adjumenta pietatis, sed etiam \emph{eficacia gratiae instrumenta}.'' ``\emph{Praestat} igitur vere Deus quicquid signis promittit ac figurat; nec effectu suo carent signa, ut verax et fidelis probetur eorum Author'').---\emph{Comment. in} Gal. iv. 9, Col. ii. 17, \emph{Inst}. iv. xiv. 17, etc. Cf. i. \emph{Conf. Helv}. c. 21, ii. c. 21. Conf. Gall. Art. 37, Catech. Gen.~v. etc. Some of the expressions in the Church Catechism, indeed, with respect to Baptism seem to Presbyterians to require at least all the explanation which Dean Goode and others have bestowed upon them. And the passage from Martensen about the ``communication of Christ's glorified corporeity'' in the Lord's Supper, which Mr.~Liddon quotes, seemingly as supplementary of the Catechism, would of course be disapproved of by Calvinists generally, although there are statements in the works of Calvin himself which might perhaps be adduced in its favour. Mr.~Liddon concludes by observing that, ``though there have been and are believers in our Lord's Divinity who deny the realities of sacramental grace, experience appears to show that their position is only a transitional one.'' There is ``a law of fatal declension,'' which will ultimately, Mr.~Liddon thinks, bring all who do not hold the High Church doctrine of the Sacraments to the Socinian position. ``Centuries,'' however, ``may intervene between the premisses and the conclusion;'' so that the prediction is a singularly safe one. By a precisely similar process of reasoning, Dr.~Manning and others are prepared to prove that there is an indissoluble connection between the worship of the virgin and a belief in the Divinity of Christ.---\emph{Engl. and Christend}. p.~civ. Faber, \emph{Growth in Holiness}, p.~72.{]}} In this respect they are similar to the other ordinances which Christ has appointed in His Church, and which by His power and Spirit are made instrumental in advancing the interests of His people. But from the very peculiarity that attaches to their distinctive character, as seals of a personal covenant between God and the believer, Sacraments may reasonably be supposed to be more effectual than non-sacramental ordinances in imparting spiritual blessings. The spiritual virtue of Sacraments is more and greater than other ordinances, just because, from their very nature, they imply more of a personal dealing between the sinner and his Saviour than non-sacramental ordinances necessarily involve.

What is the nature and extent of the supernatural grace imparted in Sacraments,---in what manner they work so as to impart spiritual benefit to the soul, it is not possible for us to define. As visible seals of God's promises and covenant, we can understand how they are naturally fitted, in the same way as the vouchers of any human engagement or covenant are naturally fitted, to attest and confirm them. But beyond this, all is unknown. The blessing of Christ and the working of His Spirit in Sacraments we cannot understand, any more than we can understand the operation of the same supernatural causes in respect of other ordinances. They have a virtue in them beyond what reason can discover in them, as naturally fitted to serve the purposes both of signs and seals of spiritual things. They have a blessing to the right receiver of them, not their own to give. ``They are made effectual means of salvation, not from any virtue in them, or in him that doth administer them, but only by the blessing of Christ, and the working of His Spirit in them who by faith receive them.''\footnote{Shorter Catechism, qu. 91.} In this respect their power and virtue are not more and not less mysterious than those of ordinances non-sacramental.

Such are the general conclusions which a consideration of the nature of the Sacraments of the New Testament lead us to acquiesce in. They are Divine institutions appointed by Christ; they are signs and significant representations of spiritual things; they are seals and vouchers of a federal transaction between God and the worthy receiver of Sacraments; they are the means for applying spiritual grace to the soul. To sum up the discussion in the language of the Shorter Catechism: ``A Sacrament is an holy ordinance instituted by Christ, wherein by sensible signs Christ and the benefits of the new covenant are represented, sealed, and applied to believers.''\footnote{Shorter Catechism, qu. 92. Calvin, \emph{Inst}. lib. iv. cap. xiv. \emph{Consensus Tigurinus} in Niemeyer's \emph{Collectio Confess}. Lipsiae 1840, pp.~192-217, translated in Calvin's \emph{Tracts}, Edin. 1849, vol.~ii. pp.~205-244. Turrettin, \emph{Opera}, tom. iii. loc. xix. qu. i.-ix. Cunningham, \emph{Works}, vol.~i. pp.~225-291, vol.~ii. pp.~201-207, vol.~iii. pp.~121-133. Amesius, \emph{Bellarm. Enerv}. tom. iii. lib. i. cap. i. Willison, \emph{Works}, Hetherington's ed. pp.~456 f. Gillespie, \emph{Aaron's Rod Blossoming}, B, iii. chap, xii.-xiv. Mastricht, \emph{Theol. Theoretico-Pract}. tom. ii. lib. vii. cap. 3.}

Sacraments and non-sacramental ordinances are like each other in two respects; and in two respects they differ. In the first place, sacramental and non-sacramental ordinances agree in this: first, that they are both positive institutions of Christ; and second, that they are both means of grace to believers. Without a Divine warrant and institution, neither non-sacramental ordinances nor Sacraments could have any place in the worship of God as part of His service; and both are therefore Divine appointments. They are both likewise means of grace to believers,---there being a positive promise attached to the right use of them, and that promise being fulfilled in the bestowment of spiritual blessing in connection with their use. This spiritual benefit, linked to the proper use of ordinances, whether sacramental or not, is over and above and quite distinct from the natural or moral influence such ordinances may have to benefit those who employ them. There is a benefit, for example, which the ordinance of preaching the Word is naturally fitted to impart, because the truth preached is adapted to man's moral and intellectual nature, and so naturally fitted to be of advantage to the hearers. In like manner there is a benefit which Sacraments are naturally fitted to impart, because they are symbolical ordinances or teaching signs; and the truths represented or taught by them are, upon the very same principle, naturally fitted to be of advantage to the receiver. But in both cases there is a blessing distinct from and additional to the natural or moral effect of the Word preached or the Sacraments administered. There is the work of the Spirit making use of Word and Sacrament to reach the understanding and the heart, and to convey to the worthy hearer or worthy receiver a spiritual blessing. And this work of the Spirit, over and above the natural effect of the truth received, is a mystery, both in the case of the ordinance of preaching and the ordinance of the Sacraments; and not, I think, a greater mystery in the one case than in the other.

We do not plead for the Sacraments as means of grace, viewed merely as natural actions and ceremonies apart from the truths which they represent, any more than we would plead for the preaching of the Word being a means of grace, viewed as the mere letter of the Word apart from the meaning of the truth which is uttered. The case of \emph{infant} Baptism, which is, as we shall afterwards see, in some respects exceptional, and not to be taken as completely bringing out the full and primary idea of the Sacrament,\footnote{Cunningham, \emph{Works}, vol.~iii. pp.~144-154.} we for the present put aside, postponing it for future consideration. But in the case of adult participation in the Sacraments, we do not plead for these generally as means of grace, when viewed simply as outward acts, and apart from the truths represented, any more than the sound of the Word preached would be a means of grace apart from the intelligent apprehension of it. Through the truths, however, in one case impressed on the hearer by significant words, and in the other case impressed on the participator through significant actions, the Spirit of God \emph{does} operate upon the intellectual and moral nature of man, making both the one ordinance and the other a means of grace. How the Spirit thus operates and imparts of His gracious gifts, we cannot tell in the one instance more than in the other. What is the mode or measure of His communications of a spiritual kind, over and above the natural or outward influence of the truth, we cannot tell. It is His own secret and supernatural work, known and recognised by the believer in the effects wrought on His soul, both in the case of the Word preached and the Sacraments administered, but not to be explained or defined in the manner of working. Let it never be forgotten that there is a mystery not to be explained whenever we get beyond the natural effect of the ordinance, whether sacramental or not, necessarily resulting from the fact that it is an effect of the Spirit, and not of any natural cause. All ordinances, as means of grace, must in that character have something in them mysterious and inexplicable. We cannot rid ourselves of the mysterious by simply ridding ourselves of sacramental ordinances,---as very many in the present day seem to imagine. We can only disconnect all mystery from the ordinances of the Church when we limit their efficacy simply to their natural influence, and deny the influence of the Spirit of God as at all connected with them.

In the second place, Sacraments differ from ordinances not sacramental in the New Testament Church, in these two things: first, they are sensible signs of spiritual truths; and second, they are seals or vouchers of a federal transaction. In respect that they are sensible exhibitions and significant actions, having a definite meaning in them, Sacraments stand out distinctly marked from other ordinances. Speaking generally, sacramental ordinances are spiritual acts of the mind or soul embodying themselves in outward and sensible actions, in so far as regards the part of the receiver in the ordinance. They are outward representations, by means of certain actions on the part of the worthy participator, of the great fact that he gives himself to Christ according to the terms of the covenant of grace. In partaking of the ordinance, he embodies in the sensible actions of the ordinance a spiritual surrender of himself to Christ, in the manner and upon the terms which Christ has appointed. This is the receiver's part in the ordinance. On the other side, Christ, through the person of the administrator of the ordinance, embodies in the actions of it a picture or representation of a spiritual communication of Himself and all the blessings of His grace to the worthy receiver. Christ, in the Sacrament, and by means of its sensible signs, gives Himself and the benefits of the new covenant, spiritually, although under an outward representation, to the believing participator. The outward signs of the Sacrament exhibit, then, a twofold action: the believer giving himself to Christ in covenant, and Christ giving Himself to the believer in the same covenant. There is a spiritual act on the part of the believer embodied in outward representation,---the act, namely, of his surrendering of himself to Christ in the way and on the terms which Christ has appointed; and there is a spiritual act on the part of Christ embodied in outward representation also,---the act, namely, of Christ with all His precious and unspeakable blessings communicating Himself to the soul of the worthy receiver. There is thus a double significance comprehended in the administration and in the participation of the sacramental ordinance, each of them having a definite and intelligible meaning of its own. In the administration of the Sacrament, Christ makes over Himself and all the benefits of His atonement to the believer, and accepts in return the believer as His. In the participation of the Sacrament on the part of the worthy receiver, he makes over himself to Christ; and receives, in return for his own soul, Christ and His covenant blessings. The double action of the administration and participation of the Sacrament is the embodiment in outward sign of a double spiritual act. There is a mutual intercommunication spiritually of Christ and the believer embodied and represented in action,---a covenant interchangeably exhibited in sensible signs, whereby Christ becomes the believer's, and the believer becomes Christ's. In their being signs of spiritual truths, Sacraments differ in a marked manner from non-sacramental ordinances.

Sacraments differ also from other ordinances in this, that they are seals or vouchers of a federal or covenant transaction. This, after all, is the grand and essential distinction between sacramental and non-sacramental ordinances. As a kind of types, as speaking and teaching signs, they are fitted to express, by the help of significant actions cognisable by the senses, the twofold spiritual act of Christ making over Himself and all His blessings to the believer, and of the believer making over himself with all his poverty and sins to Christ. But they are more than signs of a covenant thus entered into between the two parties,---they are seals and vouchers for the covenant, serving to give confirmation and validity to the engagement, as one never to be broken. In the Sacraments there is a twofold seal, as well as a twofold action, represented. There is a seal on the part of Christ, and there is a seal on the part of the believer. In marvellous condescension to our infirmity and unbelief, Christ has been pleased to add to the promise of His covenant an outward and visible voucher for it,---thereby, as it were, binding Himself doubly to the fulfilment of it, and pledging Himself, both by word and by sign, to implement all its terms. And in the worthy receiving of the Sacrament, the believer gives also a visible voucher for his part of the engagement,---thereby placing himself under new and additional obligations to give himself to Christ, and adding the outward seal to ratify the inward pledge of his heart. The covenant is mutual, and the seal is mutual. Without either part of the covenant transaction, the Sacrament would be incomplete. Withdraw Christ from the ordinance as both entering into covenant with the believer and giving him a seal of it,---take away Christ sealed to the soul in the Sacrament,---and the ordinance is reduced to a bare sign of spiritual blessing, having, perhaps, a certain natural effect by signifying truth, but empty and destitute of all spiritual grace. Or withdraw the believer from the ordinance in so far as he really by means of it gives himself to Christ,---take away the spiritual act by which the worthy participator surrenders his soul to the Saviour through his outward participation of the Sacrament,---and the Sacrament is made to be a charm, in which Christ and grace are communicated apart from the spiritual act or state of the receiver. Abstract from the ordinance the act of Christ covenanting with the believer and giving to the soul Himself and His blessings, and the remaining portion of the ordinance may continue,---the believer may still be accounted as giving himself to Christ in the Sacrament; but in the absence of Christ's act there is no spiritual blessing given in return, and the believer's act of participating in the Sacrament becomes a mere sign of adherence to Christ on his part, and nothing more than a sign.\footnote{{[}``Quod omnes fere opinantur, hoc ritu, quem Sacramentum appellant, confirmari saltem fidem nostram, ne id quidem verum censeri debet; cum nec ullo sacro testimonio comprobetur, nec ulla ratio sit cur id fieri possit. Quomodo enim potest nos in fide confirmare id \emph{quod nos ipsi facimus, quodque, licet a Domino institutum, opus tamen nostrum est?}''---Faustus Socinus, \emph{De Caena Dom, Tract. Brev}. Racovian Catechism, 1609, p.~144 f.{]}} Again, abstract from the ordinance the act of the believer spiritually covenanting with Christ and giving his soul in faith to the Saviour, and the remaining portion of the ordinance may continue,---Christ may be held as present in the Sacrament giving Himself and His supernatural grace; but in the absence of the receiver's act surrendering his soul in faith to his Saviour, the communication of spiritual grace is degraded to the position of being the result of a charm or talismanic formula,---something effected, \emph{ex opere operato}, apart from the spiritual character or faith of the receiver. It is only when the separate spiritual acts of both parties meet and combine in one transaction, that the covenant is real or complete; or that the ordinance, as a seal of the mutual engagement, is a true and proper Sacrament. As the voucher or seal of a real covenant, spiritually entered into between Christ and the believer through the ordinance, a Sacrament differs, in a very marked and important way, from ordinances not sacramental.

\hypertarget{unscriptural-or-defective-views-of-the-sacraments}{%
\section{Unscriptural Or Defective Views Of The Sacraments}\label{unscriptural-or-defective-views-of-the-sacraments}}

The principles which I have laid down in regard to the nature of Sacraments, and in regard to the difference between them and ordinances not sacramental, stand opposed to the views of two parties holding extreme positions on either side of this question. There is one party who deny the grand and characteristic distinction between sacramental and other ordinances already enunciated, and hold that the Sacraments have no virtue except as badges of a Christian profession, and signs of spiritual truths. There is another party holding opinions on the subject admitting of various modifications, but agreeing in this, that they ascribe a high spiritual efficacy to the Sacraments apart from the faith or spiritual act of the receiver. By the first party the views of the Sacraments already stated by me are held to be erroneous in the way of attributing to them a greater virtue than actually belongs to them. By the second party these views are regarded as defective in the way of ascribing to Sacraments a less virtue than really belongs to them. Let us endeavour briefly and generally to estimate the merits and truth of the principles adopted by these two parties,---reserving until a future stage in our discussions the more particular examination of their theories, in their application to the Sacraments of the New Testament individually.

\hypertarget{signs-and-no-more-than-signs-of-spiritual-things}{%
\subsection{Signs, and no more than signs, of spiritual things}\label{signs-and-no-more-than-signs-of-spiritual-things}}

The Sacraments of the New Testament are regarded by one party as signs, and no more than signs, of spiritual things,---symbolical actions fitted to represent, and impress upon the minds of men, Gospel truths. The Socinian party have made this doctrine peculiarly their own. According to their views, a federal transaction between the believer and Christ founded on His atonement is no part of the Gospel system at all; and hence the Sacraments of the New Testament can be no seals appointed and designed to ratify such a covenant. The Socinian doctrine concerning the nature of the Sacraments allows to them no more than a twofold object and design. They are not essentially distinct from other ordinances, as set apart by themselves to be the seals of the one great covenant between the believer and Christ, at his entrance into the Church at first, and from time to time afterwards, as occasion justifies or demands. But in the first place, they are signs in which something external and material is used to express what is spiritual and invisible,---the only virtue belonging to them being what they are naturally calculated to effect, as memorials, or illustrations, or exhibitions of the important facts and truths of the Gospel; and in the second place, the Sacraments are solemn pledges of discipleship on the part of those who receive them, discriminating them from other men, and forming a public profession of or testimony to their faith as Christians. These are the two grand objects, which, according to the Socinian view, the Sacraments were intended to serve; and such, according to their theory, is the nature of the ordinance.

The same system in substance, making, as it does, Sacraments entirely or essentially teaching and symbolical signs, has been adopted by many who disown the tenets of Socinianism in regard to the Gospel system generally. The theory of the Sacraments now described has been and is held by not a few in the Church of England of somewhat latitudinarian views,---the representative of such, as a class, being Bishop Hoadly. It is avowed and advocated in the present day by a very large proportion of the Independent body, who count the Sacraments to be no more than symbolical institutions, and who are ably represented by Dr.~Halley in his work, entitled, \emph{An Inquiry into the Nature of the Symbolic Institutions of the Christian Religion, usually called the Sacraments}. The single difference between the Socinian doctrine, as maintained by Socinians in the present day, and the Independent doctrine, as maintained by Dr.~Halley and others, is probably this, that Socinians limit the efficacy of the Sacraments to the natural or moral power that belongs to them as signs of Gospel truth, while Independents may admit that beyond the natural and moral power of the ordinance, as symbolical of truth, the Spirit of God makes use of them in representing truth to the mind. Let Dr.~Halley speak his own views as they are generally held by English Independents. ``The opinion we propose is, that the Sacraments are significant rites,---emblems of Divine truth,---sacred signs of the evangelical doctrine,---designed to illustrate, to enforce, or to commemorate the great and most important truths of the Gospel. Baptism, we believe, is the sign of purification, on being admitted into the kingdom of Christ, but neither the cause nor the seal of it; the Lord's Supper the commemoration of the death of Christ, the symbol of its propitiatory character, but not the assurance of our personal interest in its saving benefits. The truth exhibited in the Sacraments, just as when it is propounded in words, may be the means of the communication of Divine grace; but then the evangelical doctrine and not the Sacrament, the truth and not the symbol, the spirit and not the letter, gives life and sanctity to the recipient, as it may even to a spectator.''\footnote{Halley, \emph{The Sacraments: an Inquiry}, etc. Lond. 1844, vol.~i. p.~94 f.} According to this theory, it is the truth signified in the Sacrament---and not, over and above that, the Sacrament itself as a seal---that possesses any spiritual virtue; and that virtue may be, according to Socinians, the natural influence of the truth on the mind,---or, according to Independents, that natural influence, with the addition of the power communicated through the truth by the Spirit.

Now, in reference to this view of the Sacraments, it is necessary to bear in mind that there is no dispute as to the fact that sacramental ordinances are symbolical,---signs fitted to represent and to teach Gospel truths. Further, there is no dispute as to the fact, acknowledged by some of the advocates of this theory, that in so far as they teach or convey truth to the mind, they may be made the means of the communication of Divine grace, in the same manner very much as when the truth is propounded in words.\footnote{{[}``Es geschah in dem Zeitpunkte der Reformation, aber nicht zum ersten Male, dass die hypermystische oder zauberische Vorstellung den entgegengesetzten Fehler, die Behauptung des \emph{signum nudum} oder des blossen Bekenntnisszeichens, hervorrief. Gegen diejenige Kirche, die im Dienste der Verwandlungslehre und des \emph{opus operatum} die symbolische Natur des Sacraments verleugnete und zerstörte, hatte die sogenannte Ketzerei allezeit Recht, zunächst nur wieder das Daseyn des Symbols und die Bedeutung zu behaupten. Diejenige Kirche, die des Sacramentes Wirkung und Wesen vom lebendigen Worte und Glauben, den Sohn vom Geiste losgerissen hatte, durfte einen Gegner nie Lügen strafen oder des Unchristenthums zeihen, der der Gemeinschaft des Erlösers durch die Speise des Wortes als durch die rechte Assimilation mit seinem Leben theilhaft zu werden hoffte, und sich des Sacramentes nur noch als eines Zeichens dieser Gemeinschaft, oder auch dieses Zeichens nicht mehr bediente weil es so sehr vom Wesen abgelenkt und etwa nur habe bei noch nicht ganz befestigter Wirksamkeit des Wortes einem anfänglichen Bedürfnisse dienen sollen.'' (This is still the position of the Quakers as expounded by Barclay in his \emph{Apology}.) ``\emph{Blosse Gebetschristen}, Messalianer und dergleichen, \emph{sind nicht weniger Christen als blosse Sacramentschristen; blosse Symboliker stehen sich nicht schlechter mit der Quelle des Lebens als die Hierurgen die den Leib Christi conficiren}. Diese sind am Ende des verschwindenden Christenthums angelangt, jene stehen am Wiederanfange der Entwickelung.''---Nitzsch, \emph{prot. Beant. der Symb. Möhlers}, Hamburg 1835, p.~162.{]}} But the point in debate is, whether the Sacraments are not more than signs, and more than merely symbolical representations of truth. We hold that they are. We contend that, in addition to being signs, they are also seals,---the visible vouchers of a federal transaction between Christ and the believer who partakes of His Sacraments,---the outward pledges speaking to the eye and the senses of the completed covenant by which Christ becomes the believer's, and the believer becomes Christ's. And further, we contend that, as seals, they are made a means of grace more powerful and efficacious than simply as signs of truth.

The arguments urged by Dr.~Halley against this additional office and virtue attributed to Sacraments as more than signs, and as the seals of a federal engagement between the worthy recipient and Christ, are the two following, as stated in his own words: ``First, The ceremonial institutes of preceding dispensations, the Sacraments of the patriarchal and Jewish Church, correspond only with the view which we take of the Christian Sacraments as sacred signs of Divine truth. Second, The Sacraments considered as the causes or the means, or even the seals of converting or regenerating grace, stand opposed to the great Protestant doctrine of justification by faith without works.''\footnote{Halley, p.~95.} We shall very briefly examine each of these two objections to the view which we have announced. And we do this all the more readily, as it will afford us the better opportunity of bringing out our own principles in contrast with those embodied in the Independent theory of the Sacraments.

1\emph{st}, Dr.~Halley alleges, against the ascription to the New Testament Sacraments of the character of seals, that the ceremonial institutes of preceding dispensations, the Sacraments of the patriarchal and Jewish Church, correspond only with the views which he advocates of the Christian Sacraments as exclusively signs of Divine truth. Perhaps there never was a more unfortunate or unfounded assertion. ``One passage of St.~Paul,'' says Dr.~Halley, ``will establish this proposition.''\footnote{Ibid. p.~96.} And the single passage which is to bear the weight of the whole argument is the following one from the Epistle to the Romans: ``He is not a Jew which is one outwardly; neither is that circumcision which is outward in the flesh. But he is a Jew which is one inwardly; and circumcision is that of the heart, in the spirit, and not in the letter; whose praise is not of men, but of God.''\footnote{Rom. ii. 28, 29.} This is the solitary passage quoted to prove the broad and general assertion, that the Sacraments of the patriarchal and Jewish Church afford no precedent or example of Sacraments as seals, but only of Sacraments as signs. The verses quoted plainly amount to nothing more than a statement of the difference between what the apostle calls circumcision outwardly and circumcision inwardly, the external rite and the internal grace, and a declaration that a man might have the outward rite, and not the inward grace. The apostle does not say, and cannot, except by a violent misapplication of his words, be made to say, that \emph{in the case of the man who has both the outward and inward circumcision}, the external rite may not be the visible seal of the spiritual grace. The very opposite of this the same apostle in the very same Epistle undeniably asserts. In language as plain as he could possibly select or employ, Paul affirms that in the case of Abraham, who had the inward grace, the outward rite of circumcision was a seal to him of that grace. ``Abraham,'' says the apostle, ``received the sign of circumcision, a \emph{seal} of the righteousness of the faith which he had yet being uncircumcised.''\footnote{Rom. iv. 11.} And how is it that Dr.~Halley gets rid of this express assertion of the apostle, standing as it does in explicit contradiction to his general averment that the Sacraments of the Jewish Church were signs and not seals? He admits that to Abraham personally and individually circumcision was a seal, and not merely a sign. But by a strange misapprehension of the doctrine of his opponents, he argues that it could not be a seal of faith to others of Abraham's family or countrymen who had not his faith. ``Although,'' says Dr.~Halley, ``to him circumcision was the seal of faith, it could not have been so to his posterity.'' ``Was it,'' he asks, ``was it, in this sense, a seal of the righteousness which they had, an approval of their faith, to the men of his clan, or to Ishmael, or to the infants of his household, or to any of his posterity in subsequent ages?''\footnote{Halley, p.~100.} The answer to such a question is abundantly obvious. If the men of Abraham's clan had not faith, if Ishmael had not faith, circumcision could have been no seal of faith to them. The outward rite could not be a seal of the inward grace, when the latter did not exist. It could not be a seal of a spiritual covenant between them and God which had not been entered into. I do not stop to consider the question of whether or not circumcision is to be accounted, even in such a case, the seal to such individuals of the outward blessings promised to them, as Jews, by God, as the rightful King of Israel as a nation; but, as a seal of a spiritual covenant, it of course could not be a seal at all to those who were not parties to the covenant,---while it was a seal, according to the explicit assertion of the apostle, to those who were. The very express statement of Paul cannot be evaded, but fully bears out the assertion that the Sacraments of the Jewish Church were not signs alone, but seals of a spiritual covenant to those who were really parties to the covenant. ``Abraham received the sign of circumcision, a \emph{seal} of the righteousness of the faith which he \emph{had}.''

2\emph{d}, Dr.~Halley alleges that the Sacraments, if they are considered as the cause or the means, or even the seals of spiritual and saving grace, would be opposed to the great Protestant doctrine of justification by faith without works. Now it is readily admitted, that if Sacraments are regarded as the causes or means of justification, they are utterly inconsistent with the Protestant doctrine of justification by faith alone; and in this point of view the objection is true and unanswerable when directed against some of those theories of the Sacraments which we may be called upon to consider by and by. But it is denied that the objection is true when directed against the theory of the Sacraments which maintains that they are not causes and not means of justification, but seals of it and of other blessings of the new covenant. The Sacraments as seals, not causes of justification, cannot interfere with the doctrine of justification by faith, for this plain reason, that before the seal is added, the justification is completed. The seal implied in the Sacrament presupposes justification, and does not directly or instrumentally cause it; the seal is a voucher given to the believer that he is justified already, and not a means or a cause of procuring justification for him. Justification exists before the seal that attests it is bestowed. The believer has previously been ``justified by faith without the works of the law,'' ere the Sacrament of which he partakes can affix the visible seal to his justification. All this is abundantly obvious; and the objection of Independents, that the doctrine of the Sacraments as personal seals is opposed to the principle of justification by faith, is wholly without foundation. That the Sacraments are a means of grace additional to what the believer possessed before his participation in them, it is not necessary to deny, but rather proper strongly to assert.\footnote{{[}``Das Gläubigste, so zu sagen, am sacramentlichen christlichen Gemeinglauben ist doch wohl dieses: je mehr das Sacrament mit voller Empfänglichkeit genossen wird, desto weniger ist es blosses Zeichen, oder blosses Unterpfand der Lebensmittheilung Christi, desto mehr diese Mittheilung selbst. Das Sacrament ist Leiter, Kanal der Gnade, wie der römische Katechismus sich ausdrückt. Bis auf diesen Punkt wird der Sacramentsbegriff---ich will zugeben unter sehr verschiedenen Bedingungen (from those of the Romanist theory)---durch das in dieser Hinsicht ganz ungetheilte Bekenntniss der Protestanten gesteigert. . . . Der protestantische Begriff des Siegels oder Pfandes ist weit entfernt die \emph{collative} Kraft des Sacraments zu schwächen; er gestattet sogar die mystische Verknüpfung der Elemente des Sacraments mit der \emph{res signata et exhibenda; signa et res significatae sacramentaliter conjunguntur} (\emph{Conf. Helv. post}. xix.). \emph{Bezeichnung, Besiegelung, Darreichung} der Gnade Christi vereinigen sich im Sacramente. (\emph{Decl. Thorun. De Sacr}. 1, 7).''---Nitzsch, \emph{prot. Beant, der Symb. Möhlers}, Hamburg 1835, p.~151 f.{]}} In entering into a personal covenant with Christ through particiption in the Sacraments, or in renewing that covenant from time to time, the faith of the believer is called forth and brought into exercise in the very act of participation, and by the aids to faith which the ordinance affords. And in answer to this faith so exercised and elicited, there is an increase of grace given to the worthy recipient above and beyond what he had before. The faith of the believer, called into exercise in partaking of the ordinance and by means of it, is met by the bestowment of corresponding grace. But it is never to be forgotten that the Sacraments presuppose the existence of grace, however they may give to him that already has it more abundantly. They presuppose, and beforehand require, that a man is justified by faith before they give their seal to his justification.

There is no ground, then, in Scripture, but the very opposite, for asserting that the Sacraments are no more than signs or symbolical actions, as held by Dr.~Halley and those whom on this question he represents. The fundamental error involved in the views now adverted to is, the denial of Christ's part in the federal transaction involved in a Sacrament. Independents overlook \emph{His} department of the work in the engagement entered into through means of the act of receiving the Sacraments; and in the absence of the act of Christ giving Himself and all His spiritual blessings to the believer in the ordinance, the act of the recipient is not met by the grace that Christ confers, but is reduced to a mere significant dedication of himself to the Saviour unconnected with any grace at all. Take away Christ from the ordinance as present there, to covenant with the believer, actually giving Himself and His blessings spiritually through means of the outward ordinance, in answer to the faith of the believer giving himself to Christ through the same ordinance, and the Sacrament is evacuated of all spiritual grace; the act of the receiver becomes a mere expressive sign of what he is willing to do in the way of dedicating himself to Christ; but not an actual dedication, accomplished through means of a covenant \emph{then} and \emph{there} renewed, by which the believer becomes Christ's, and Christ becomes the believer's. The principle of the Independents in regard to the Sacraments cuts the Sacrament, as it were, in twain, and puts asunder what God has joined. It leaves to the believer his part in the transaction, in so far as he employs the Sacrament as a sign of his dedication to Christ; but it takes away Christ's part in the transaction, in so far as He meets with the believer and enters into covenant with him,---accepting the believer as His, and giving Himself to the soul in return. Severed from Christ in the ordinance, and from the covenant with His people into which Christ there enters, the act of the recipient can be no more than an expressive sign, or convenient profession of faith, unconnected with true and proper sacramental grace.

\hypertarget{effectual-to-impart-justifying-and-saving-grace-directly}{%
\subsection{Effectual to impart justifying and saving grace directly}\label{effectual-to-impart-justifying-and-saving-grace-directly}}

The Sacraments of the New Testament are regarded by another party as in themselves, and by reason of the virtue that belongs to them, and not through the instrumentality of the faith or the Spirit in the heart of the recipient, effectual to impart justifying and saving grace directly, in all cases where it is not resisted by an unworthy reception of the ordinance. This general opinion may be held under various modifications; but all of them are opposed to the doctrine I have already laid down, that the Sacraments are seals of a justifying and saving grace already enjoyed by the recipient, and not intended for the conversion of sinners; and that they become means of grace only in so far as the Spirit of God, by the aid of the ordinance, calls forth the faith of the recipient, and no further.

The doctrine of the efficacy of Sacraments, directly and immediately of themselves, and not indirectly and mediately through the faith of the receiver, and through the Spirit in the receiver, is advocated in its extreme and unmodified form by the Church of Rome. According to that Church, these ordinances, as outward and material rites, become, after certain words of institution pronounced by the priest, possessed of a sacramental virtue, which is conveyed infallibly to the soul of the person who receives them, on \emph{two} conditions, which are necessary to justifying and spiritual grace being really imparted. \emph{First}, on the side of the priest who pronounces the words of institution, there is required, as a condition of the supernatural grace being imparted, that he have the intention to make the Sacrament and confer it; for without this, the outward matter of the ordinance would remain mere matter, and have no sacramental character or virtue. And \emph{second}, on the side of the recipient of the ordinance, it is required that he be free from any of those sins which, in the language of Popery, are called ``mortal,'' and which, when contracted and not removed, would resist the operation of the sacramental virtue, and prevent his soul receiving spiritual grace. But when these two conditions are present,---when the priest intends to consecrate and dispense the ordinance, and the recipient is not barred from the reception of its virtue by mortal sin,---such is the efficacy of the Sacrament in itself, and directly, that it infallibly communicates to the partaker of it justifying and saving grace. The doctrine of the Church of Rome is very distinctly brought out in the canons of the Council of Trent, and also in her Catechism. ``If any,'' says the 11th canon concerning the Sacraments in general, ``if any shall say that there is not required in the ministers, when they make and confer the Sacraments, at least the intention of doing what the Church does, let him be accursed.'' ``If any shall say that the Sacraments of the New Law do not contain the grace of which they are the signs, or that they do not confer that grace on those who place no obstacle in the way, as if they were only outward signs of grace or justification already received by faith, and certain badges of the Christian profession, by which believers are distinguished from infidels, let him be accursed.'' ``If any shall say that grace is not conferred by the Sacraments of the New Law, \emph{ex opere operato}, but that faith in the Divine promise alone avails to secure grace, let him be accursed.''\footnote{\emph{Concil. Trident. Canones et Decreta}, Sess. vii. \emph{De Sacr. in Gen.}, Can. vi. viii. xi. {[}Compare Möhler's statement of the Roman Catholic theory of the Sacraments (\emph{Symb}. 6te Aufl. pp.~253-258). It is worthy of remark, that that acute and dexterous controversialist, in his own exposition of the doctrine, passes over in utter silence the very remarkable and important element of the priest's intention, as defined at Trent,---the sole reference to it being in a quotation from Bellarmine given in a note (p.~256). Nitzsch's remarks on the significance of the point thus ignored, are worth quoting: ``The demand of an \emph{intention} on the part of the priest in order to the Sacrament being savingly effectual, or effectual at all, met opposition among the Romish theologians, both at and before the Council of Trent. At one time the danger was pleaded incurred by the baptized or the absolved, who might now so easily miss obtaining grace, or be left in uncertainty about it; at another, the much greater concession already made, that unbelief, or even mortal sin on the part of the priest, did not destroy the efficacy of the priestly act. On these grounds, the Council saw themselves compelled to restrict the demand as much as possible; it asserted itself, notwithstanding. Sess. vii. Can. 11: `Si quis dixerit in ministris, dum Sacramenta conficiunt et conferunt, non requiri intentionem saltem faciendi quod facit Ecclesia, anathema sit,' with good reason, as is easily to be seen. For if, as Can. x. decided, the private Christian cannot make or confer most of the Sacraments,---if the supernatural qualification of the priest, although a `\emph{gratia gratis data},' and not `\emph{gratum faciens},' still was of the essence of the sacramental dispensation,---while, on the other hand, no `\emph{bonus motus}' of the recipient was necessary for the reception of the grace, nay, in the case of private mass, no recipient at all, in the case of infant Baptism, no conscious recipient, was needed,---then, should there be an \emph{entire} absence of harmony between the mental state of the priest and the design of the transaction, there would be absolutely nothing left but the bare, mechanical, accidental, external act; and from this hardly a single believer would have expected any blessing whatsoever. The worth of the words of institution and promise, as appropriated by faith, had already been sacrificed to the worth of the `\emph{opus;}' so had the dignity of the congregation to the dignity of the priest: hence arose great perils and perplexities, if now, after all, the worth of the words should be allowed to stand without moral desert on the part of the priest. The morally indifferent supernatural qualification of the priest must therefore now receive at least a psychological quickening, and the co-operation of the mental state of the priest be thus brought in to give the requisite support to a transaction otherwise bereft of all substance and security. They resigned themselves, accordingly, to the lesser perplexity. The doctrine of the `\emph{intentio ministri}' is a reinforcement to the doctrine of the `\emph{opus operatum},' which yields at the same time various advantages of another sort; and the latter dogma is again explained and supported by the notion of the `not interposed hindrance.'\,'' \emph{Prot. Beant}. Hamburg 1835, p.~154. Gerhard, \emph{Loci}, xviii. 31-38, ed. Preuss. tom. iv. pp.~151-158.{]}} According to this doctrine, then, Sacraments impart grace, not through the channel of the faith of the receiver, and not in dependence in any way on his spiritual act, but immediately and directly from themselves, ``\emph{ex opere operato}.'' This last expression is to be interpreted in connection with the distinction drawn by the Church of Rome between the Sacraments of the Old and New Testament Churches. The Sacraments of the Gospel Church are superior in efficacy to those under the law, in the Popish theory, because the former, or the New Testament Sacraments, work grace independently of the spiritual disposition or act of the recipient; whereas the latter, or Old Testament Sacraments, were dependent on the spiritual disposition or act of the receiver of them. The ``\emph{opus operatum}'' of the New Testament Sacraments, or the virtue they have by their own act, apart from the spiritual state of the recipient, is contrasted with the ``\emph{opus operantis}'' in the Old Testament Sacraments, or the virtue which they had, not in themselves, or in their own operation, but only in connection with the spiritual act of the partaker. According to the proper theory of the Church of Rome, the Sacraments of the New Testament impart grace \emph{ex opere operato}, or from their own intrinsic virtue and direct act on the soul of him who receives them.\footnote{The statement that Papists hold the Sacraments to be efficacious of themselves, apart from the spiritual condition of the recipient, is often met---especially by English Romanists---with a flat denial. And on this ground. They hold that many elements are, in point of fact, present in every case in which the Sacraments are efficacious; some of these elements are connected with the state of the recipient,---such as a desire to receive the ordinance,---and others with the working of God. Thus Bellarmine objects to Calvin's stating the point in debate to be, not as to grace being conferred in the Sacraments, but only ``whether God works in them by His own proper, and, so to speak, intrinsic virtue, or whether He resigns His place to the external signs.''---\emph{Inst}. lib. iv. c. xiv. 17. (See next note.) Any Romanist, however, who has the slightest regard for the authoritative declarations of his Church, can be fixed down conclusively to this position, that, whatever other elements may, in point of fact, be present, the immediate, efficient, instrumental cause of the grace invariably conveyed to all ``qui non ponunt obicem'' is ``the outward action called a Sacrament,'' and nothing else. See Turrettin's masterly treatment of this point, \emph{Opera}, loc. xix. Qu. viii. 2-6. Cunningham, \emph{Works}, vol.~iii. pp.~124-139. Hodge, \emph{Princeton Essays and Reviews}, New York 1857, pp.~370 f., 388.}

This doctrine of the inherent power of Sacraments in themselves to impart grace, held by the Church of Rome, is also the system maintained, although with some important modifications, by another party beyond the pale of that Church, the representatives of which, at the present day, are to be found in the High Churchmen of the English Establishment. The doctrine of the High Church party in the English Establishment in regard to the Sacraments differs indeed in \emph{two} important particulars from the full and unmodified development of it found in the Popish system; but in other respects it is substantially the same,---equally implying the inherent power of Sacraments to impart grace, not through the spiritual act of the recipient, but apart from and independently of it. The advocates of High Church principles in the Church of England generally---although there is a numerous and increasing section of them who in this respect approximate more nearly to Rome---generally reject the Popish doctrines,---\emph{first}, of the \emph{opus operatum}, and \emph{second}, of the necessity for the intention of the priest in the Sacrament. They deny that the Sacraments have any immediate physical influence upon the soul, by the very act of outwardly participating in them,---such as is implied in the \emph{opus operatum} of the Church of Rome; and they deny, further, that the intention of the priest to make and confer the Sacrament is a necessary condition of it, without which it could impart no grace. These two elements in the Popish theory of sacramental ordinances are rejected, generally speaking, by the High Church disciples of the English Establishment, although instances are not awanting---and they seem to be multiplying of late---of both these monstrous pretensions being, in a certain sense, maintained by them. But they agree with the Romish Church in the grand and fundamental principle which belongs to its doctrine of the Sacraments,---namely, that they communicate grace from the sacramental virtue that resides in themselves,---or, as some prefer to put it, that invariably accompanies them by Christ's appointment,\footnote{In, cum, or sub Sacramento. {[}``It is to be observed,'' says Bellarmine, ``that the dispute is not about the \emph{mode} in which Sacraments are causes of justification, \emph{i.e}. whether the effect is produced by physical or moral means; and again, if the influence be of a physical sort, whether it be by some inherent quality, or by the simple will of God; for these points do not belong to the question of faith; but only in general, whether the Sacraments are true and proper causes of justification; so that it truly follows, from a man's being baptized, that he is justified.''---\emph{Disputationes}, tom. iii. lib. ii. cap. i.{]}}---and by their own immediate influence on the soul, and not instrumentally by the operation of the Spirit of God on the worthy recipient and through the medium of his faith. This is the characteristic principle that is common both to the Popish and the High Church theories of Sacraments. Both these parties hold that there is something in or connected with the ordinance which directly and immediately does the work of grace upon the soul; and not merely indirectly and mediately through the Spirit of God working on the soul, and the faith of the soul working in return. The Church of Rome ascribes this efficacy of the ordinances to the \emph{opus operatum} of the Sacraments, and the act and intention of the priest in consecrating them. The High Churchmen of the English Establishment usually reject both of these doctrines as laid down by the Council of Trent, and ascribe this efficacy of the ordinances to the deposit of spiritual grace which Christ has communicated to the Church, and connected with the Sacraments, and given them the power to impart. But the High Churchmen of Rome and the High Churchmen of England agree equally in this, that there are in the Sacraments an efficacy and power to impart grace of themselves, directly and immediately, to the soul of the recipient; and that they are not merely aids or instruments for bringing the recipient into direct and immediate communication with Christ to receive grace from Him.\footnote{{[}Goode, \emph{Nature of Christ's Presence in the Eucharist}, Lond. 1856, vol.~i. pp.~vi. 11-55. Cunningham, \emph{Works}, vol.~i. pp.~233-237.{]}}

Although both the Canons and Catechism of the Council of Trent lay down, to all appearance, expressly and undeniably the doctrine that there is a physical virtue in Sacraments, whereby they operate upon the recipient, yet there are not awanting doctors of the Romish Church who are anxious to soften down the dogma of the \emph{opus operatum}, and to explain it in the sense of a moral and spiritual, and not a physical virtue, residing in the ordinance. And in this modified form of it, the Romish doctrine of the Sacraments---apart from the necessity of the priest's intention---approximates very closely to the High Church theory entertained by many in the Church of England. That theory maintains the doctrine of not a physical but a spiritual virtue deposited and residing in the Sacrament, which operates universally, not through the faith or spiritual act of the recipient, but directly and immediately through the act of participation in the outward ordinance. This, in fact, is no more than part of the general doctrine that the Church is the grand storehouse of grace to man, and not Christ Himself; and that it is by communication with the Church, and not by direct communication with Christ, that the soul is made partaker of that grace. The Sacraments, as the chief medium through which the Church communicates of its stores of spiritual blessings, are the efficient instruments for imparting grace directly to the recipient.

Now, there is one preliminary remark which, in proceeding to estimate the value and truth of such principles in regard to the Sacraments, it is necessary to bear in mind. It is not denied, but, on the contrary, strongly maintained and asserted, that the Sacraments are means of grace. To the believer who uses them aright, they are made the means of conveying spiritual blessings. In regard to this, there is no controversy between the opponents and the advocates of High Church views of the Sacraments, whether Popish or Tractarian. But the question in dispute is, whether the Sacraments become effectual, from a virtue in themselves, or in the priest that consecrates them, or only by the work or the Spirit and the faith of the recipient? That the faith of the believer is called forth and exercised in the ordinance, and that through this faith he receives grace additional to what he enjoyed before, we do not dispute, but, on the contrary, strenuously maintain. That the spiritual act of the believer in the ordinance, when in faith he gives himself to his Saviour, is met by the spiritual act of Christ in the ordinance, when in return He gives Himself and His grace to the believer, is a doctrine at all times to be asserted and vindicated. That the faith of the recipient, in the act of committing and engaging himself to Christ, through means of the ordinance, is a faith unto which Christ is given in return, we would constantly affirm; and in this sense, and in this way, the Sacraments become means or channels or instruments whereby grace is given and conveyed. But they are no means of grace except through the faith of the recipient, and in consequence of his own spiritual state and act. There is no inherent power in the ordinance itself to confer blessing, apart from the faith of the participator, and except through the channel of that faith. There is no deposit of power---whether, with the Church of Rome, we deem it physical and \emph{ex opere operato}, or whether, with Tractarians and High Churchmen, we call it spiritual---in the Sacraments themselves to influence the mind of him who receives them. They have no virtue of themselves, apart from the work of Christ through His Spirit on the one side, and the spiritual act of the recipient through his faith on the other side. In the language of Amesius, in his admirable reply to Bellarmine, Sacraments have no power ``efficere gratiam immediate, sed mediante Spiritu Dei et fide.''\footnote{Amesius, \emph{Bellarm. Enerv}. Amsterdam 1658, tom. iii. lib. i. cap. v. p.~22. ``In the following sentences of the \emph{Declaration of Thorn},'' observes Nitzsch, ``all Protestants agree: `Sacraments are outward and visible signs, seals, and testimonials of the Divine will, instituted by God Himself, by the combination of Word and element, in order to seal and exhibit, through means of these signs, the invisible grace which is promised in the Word of the covenant. It is obvious that we by no means make them bare signs, empty and inefficacious, or mere badges of outward profession, since, besides their mystical significance, according to the Divine institution, we attribute to the Sacraments a sure sealing of God's promises, and at the same time a true and infallible exhibition of the things promised, in the way suited and proper to them, to be received by a living faith.'\,'' (Niemeyer, p.~680.) \emph{Prot. Beant. der Symb. Möhler's}, Hamburg 1835, p.~175. Bruce, \emph{Serm. on the Sacr}. Wodrow Soc. ed. Edin. 1843, p.~10 f. Calvin, \emph{Antidote to Council of Trent}, Sess. vii. Can. ii. iv.-vi. \emph{Tracts}, vol.~iii. Calvin Transl. Soc. Edin. 1851, pp.~172-175.{]}}

Has the Church, then, ordinances for its administration and use which, either by the original appointment of Christ, or by deposit of grace from Christ, have in themselves virtue to impart spiritual blessing through the administration of them alone? Or has the Church ordinances for its administration and use which have no virtue in themselves to communicate grace, except in connection with the faith of the receiver, and the blessing imparted by the Spirit? Are the Sacraments of the New Testament themselves a quickening power in the soul, apart from the faith or spiritual act of the participator,---the original deposit of grace committed to them being still retained, and still communicable through their administration, and that alone? Or are these Sacraments effectual to impart grace only in connection with the faith and spiritual disposition of the recipient,---there being necessary to their efficacy, both the act of the believer, in the use of them, giving himself to Christ, and the act of Christ, through the same ordinance, giving Himself to the believer. It matters little whether, as with the Popish Church, the Sacraments are invested with a \emph{physical} virtue, in consequence of which they impart grace; or whether, as with the High Churchmen of other denominations, they are invested with a \emph{spiritual} virtue in consequence of which they impart grace,---if in both cases the grace is given by the Sacrament itself, and not given through the Spirit and the faith in the heart of the recipient. It matters little whether a physical or a spiritual explanation is given of sacramental efficacy, if it be efficacy exerted apart from Christ in the ordinance giving Himself to the believer, and experienced apart from the believer in the ordinance giving himself to Christ. Whatever be the efficacy and virtue, physical or moral, if it is independent of and separate from the faith of the recipient covenanting in the ordinance with Christ, and the act in answer to that faith of Christ covenanting with the recipient, it is not the sacramental grace which the Scripture recognises. It becomes, when thus separated and drawn apart, a mere charm, a trick of magic, whether physical or spiritual, utterly unknown to the Gospel economy. Let us endeavour to apply to this theory those tests which may serve to try its merits and its truth. There are four different tests by which we may try the merits of this sacramental theory, whether held in its extreme form by Papists, or in its more modified form by High Churchmen of other communions.

1\emph{st}, Tested by Scripture, which constitutes the rule for the exercise of Church power, there is no warrant for asserting that there is an inherent and independent virtue in Sacraments to impart justifying or saving grace.

The truth of this general proposition may be established by a very wide and ample deduction of evidence from Scripture. It is impossible for us to do more than advert to the leading heads of proof in connection with this question. \emph{In the first place}, those multiplied and various declarations of Scripture, which state that we are justified by faith alone without works on our part, very distinctly prove that the Sacraments cannot have an independent and inherent power in themselves of conveying justifying and saving grace. Such passages expressly assert that faith is the immediate instrumental cause of justification. They are inconsistent, therefore, with the theory that the Sacraments directly and immediately of themselves impart grace, although they are quite consistent with the doctrine that the Sacraments indirectly, and through the faith of the worthy receiver, may impart grace. \emph{In the second place}, the doctrine that the Sacraments have an inherent virtue to confer grace, is opposed to the whole tenor of Scripture, which sets forth Christ as the one and the immediate object of faith and hope to the believer, in the matter of his justification and salvation. The Word of God, from its commencement to its close, clearly and constantly and invariably points to Christ, and to nothing but Christ, as the only source to which a sinner must look for forgiveness and acceptance with God. The theory of the Sacraments held by High Churchmen presents another and a different object for his faith, and teaches him to rest in an outward observance as sufficient. It is part of that most destructive system which places the Church and the ordinances of the Church between the sinner and his Saviour. \emph{In the third place}, the very express testimony of the Apostle Paul, in regard to the insufficiency of the Sacraments under the Old Testament Church to communicate grace of themselves, is an argument equally effectual to show that the New Testament Sacraments are insufficient likewise. Abraham was not justified by circumcision, but by the faith of which his circumcision was the seal.\footnote{Rom. ii. 25-29, iii. 20, 30, iv. 3-11; Heb. ix. 11 f., x. 1-11. {[}Comp. the Apology for the Confession of Augsburg, vii. 18, p.~203, in Hase, \emph{Libri Symbolici Eccles. Evang}. Lipsiae 1827.{]}} \emph{In the fourth place}, the statements of Scripture which at first sight might be construed as if they ascribed a gracious influence to the Sacraments of the New Testament in themselves, and which seem to connect saving benefits with the observance of them, are not stronger or more numerous, but less so, than those which ascribe justifying and saving blessings to the ordinance of the Word, or truth received by the reader or hearer of it. We know that the Word or the truth justifies, not of itself, but through the faith of him that receives it; and that, apart from this faith, it has no virtue or power of a gracious kind at all. In the same manner, Sacraments impart grace, not of themselves, but through the faith of those who receive them; and, apart from that faith, they have no life or blessing whatsoever. \emph{In the fifth place}, the theory of an inherent virtue or power in the administration of the outward ordinance is utterly opposed to those numerous passages of Scripture which assert that the power of the Gospel is altogether of a spiritual kind, and is in no respect akin to a mere external and material influence, as if such could impart a supernatural grace. It is ``not meat and drink, but righteousness, and peace, and joy in the Holy Ghost.'' And instead of pointing to any outward source of power or efficacy, and exclaiming, ``Lo here, or Lo there!'' the Christian has been taught to think that ``the kingdom of God'' has its source and presence ``within him.''\footnote{Rom. xiv. 17; Luke xvii. 21.} The theory which ascribes to the Sacraments an infallible virtue which, unless counteracted by some obstacle, such as infidelity or open vice, must operate to impart grace, is inconsistent with those numerous statements of Scripture which represent the Gospel as a spiritual power, adapted to the spiritual nature of man.\footnote{Gillespie, \emph{Aaron's Rod Blossoming}, B. iii. chap. xii.-xiv.}

In estimating the bearing of Scripture testimony on this question, there is one consideration of a general kind which it is of great importance to the argument to bear in mind. In every theory of the Sacraments that can be held,---from the lowest to the highest, from the Socinian up to the Popish,---the Sacraments are regarded as at least signs of spiritual things, representing and exhibiting the blessing in outward resemblance. The union thus established, according to any theory that can be held of them, between the sign and the thing signified by it, has introduced into Scripture a kind of phraseology which at first sight appears to give some sanction to the High Church system in regard to sacramental ordinances. There is often an exchange of names between the sign and the thing signified in Scripture, in consequence of which what may be predicated of the one is often asserted of the other, and \emph{vice versâ}. This usage of language, so frequently exemplified in Scripture in connection with this matter, is a usage found commonly in other writings and in regard to other matters, and gives rise to no sort of misapprehension in our interpretation of it. It is the great foundation indeed of all figurative language.\footnote{{[}``Omnia significantia videntur quodam modo earum rerum quas significant sustinere personas: sicut dictum est ab Apostolo, `Petra erat Christus,' quoniam petra illa, de qua hoc dictum est, significabat utique Christum.''---Aug. \emph{De Civitate Dei}, lib. xviii. cap. 48.{]}} Thus, when Christ is said to be ``the Passover sacrificed for us,'' there is an exchange of this kind, in which the name of the sign is given to the thing signified; and when Christ says of the bread, ``This is my body,'' there is an exchange in the opposite way, and the name of the thing signified is attributed to the sign. And in perfect accordance with this usage of language, there are several passages in Scripture in which the mere outward observance in the case of the New Testament Sacraments, the external sign, has a virtue attributed to it which in reality belongs, not to the sign, but to the grace represented in the observance, or to the thing signified. Thus, for example, ``\emph{Baptism}'' is said in one passage ``\emph{to save us};'' although, from the further explanation contained in the passage itself, it is plain that it is not the outward sign but the thing signified that is spoken of under the name of the sign; for the apostle adds immediately, ``not the putting away of the filth of the flesh, but the answer of a good conscience towards God.''\footnote{1 Pet. iii. 21.} In the same manner the Apostle Paul speaks of ``the cup of blessing'' as ``the communion of the blood of Christ,''\footnote{1 Cor. x. 16.}---language in which that is predicated of the sign which is truly predicated only of the thing signified. In short, the sacramental union between the outward sign and the inward grace gives occasion to not a few examples in Scripture in which what is true of the one only, or the inward grace, is attributed to the other, or the outward sign. Almost the whole plausibility of the argument from Scripture in favour of the High Church theory of the Sacraments comes from this source; and it is completely removed when the familiar canon of criticism, applicable to Scripture in common with other writings, is attended to,---namely, that what truly belongs to the thing signified is often predicated figuratively of the sign, and so ought to be interpreted and understood.\footnote{{[}Westminst. Conf. chap. xxvii. 2, xxix. 5. Goode, \emph{Nature of Christ's Presence in the Eucharist}, Lond. 1856, vol.~i. pp.~241-250, 598.{]}}

2\emph{d}, The theory of an inherent power, physical or spiritual, in the Sacraments, is inconsistent with the supreme authority of Christ, from whom all Church power is derived.

The doctrine that would deposit in sacramental ordinances a grace communicable to the participator, apart from his communion with Christ, directly and immediately, is inconsistent with the office and right of Christ to hold in His own hand all blessing, and to dispense from His own hand, not mediately through another, but at once from Himself, the grace which His people receive. Such a theory takes the administration of grace out of the hands of Christ, ever present to dispense it, and transfers it to a priest standing in His room. There can be no participation in heavenly blessing except what comes from direct communication with Christ on the part of the soul that receives it; and it is a dishonour to Him, who is the ever-living and ever-present administrator of all grace to His people, to put the mute and conscious ordinance in the place of Christ, and to transfer the dependence of the soul for spiritual blessing from the Divine Head in heaven to the outward ministry of Sacraments on earth. That Christ \emph{might} by His original appointment have made the Sacraments the receptacle of a physical influence, fitted and able to work a supernatural blessing on the soul, it would perhaps be presumptuous to deny. That Christ \emph{might} at the first institution of the ordinances have made them a reservoir or storehouse of grace enough for all ages of the Church, and imparted to them a spiritual blessing out of which every subsequent generation of His people might draw their supply, we need not be anxious to dispute. Or that Christ, without communicating at the beginning to Sacraments a store either of physical or spiritual grace sufficient for all generations, might have tied Himself up to the indiscriminate and invariable communication of His Spirit along with the administration of outward Sacraments, and bound Himself down, without any choice or discretion, to link spiritual grace to material rites, apart from the faith of the person observing them,---this, too, is perhaps a \emph{possible} imagination. But had Christ, as the Head of ordinances in His Church, done either the one or the other of these things, He must to that extent have divested Himself of His office as Mediator, or resigned the exercise of it; He must in so far have abdicated His functions as the sole and living and ever-present administrator of grace to His Church; and been shut out from that exclusive and supreme agency which He maintains as the dispenser as well as author of every blessing by which the soul is to be saved.

3\emph{d}, The theory of the Sacraments which ascribes to them an independent virtue or power, is inconsistent with the spiritual liberties of Christ's people.

Such a system brings the soul itself into bondage. It keeps the spirit, which Christ has Himself redeemed, waiting upon man for the communication of the blessings of its redemption; it makes the soul which Christ has ransomed dependent for its freedom on the ministry of a fellow-creature. There cannot be a worse or more abject thraldom than that which subordinates the flock of the Saviour's purchase to any one but Himself, and causes them to hang upon the intention entertained or not entertained by a priest for the enjoyment or forfeiture of spiritual blessing. But even apart from the monstrous doctrine of the Romish Church as to the intention of the priest being necessary to the efficacy of the ordinance, the sacramental theory we have been considering, whether Popish or Tractarian, is inconsistent with the spiritual freedom of those whom Christ has redeemed. That freedom consists in subjection to and dependence on Christ, and none but Christ,---in being emancipated from all dependence on any other except their Saviour,---in being kept waiting, not at the footstool of man for saving blessings, but at the footstool of Christ,---and in being taught to look for all the grace they need day by day, not to the ministry of man's hand, but to the hand of Christ. Spiritual freedom for the believer is bound up with a dependence on Christ immediately and directly, and on Him alone, for every blessing that he needs.\footnote{{[}Goode, \emph{Letter to a Lay Friend}, Lond. 1845, pp.~18-24. Litton, \emph{Church of Christ}, Lond. 1851, pp.~11-13, 202-232, 240 f.{]}}

4\emph{th}, The sacramental theory we have been considering is inconsistent with the spirituality of the Church, and of the power exercised by the Church for the spiritual good of men.

When, according to that theory, the Sacraments become the instruments of justification and the source of faith, instead of the seal of a justification already possessed, and the exercise and aid of a faith already in existence,---when they are made to come between the soul, in its approach to Christ, and Christ Himself, and communion in the external ordinance is substituted for the fellowship of the Spirit, it is a fatal evidence that the Church, which so teaches and so practises her teaching, although she has ``begun in the Spirit,'' has ``sought to be made perfect by the flesh.''\footnote{Gal. iii. 3.} If the external ordinance be made to occupy that place which belongs to the Spirit, and participation in the ordinance be the substitute for faith, the sacramental theory thus reduced to practice will be but the commencement of worse and deeper degradation. It is but the beginning of a course which, consistently followed, must lead to a religion of form and self-righteousness, of sense and sensuous observances, of carnal ordinances and a ceremonial holiness, of outward satisfaction and penances and merit. There will be the priest and the bloodless but efficacious sacrifice, grace conferred by the tricks of a physical or spiritual magic, a religion that manifests itself outwardly and not inwardly, the holiness of houses, and altars, and sacred wood and stone, but not the holiness of the Spirit; the atonement of Sacraments and penances and creature merits, but not the atonement of the Saviour received by faith; a righteousness of bodily discipline and fleshly mortification, but not the righteousness of God imputed to the believer; a justification made out of pains and merits, of sufferings and works, but not a justification freely given by Divine grace and freely accepted by faith; an outward baptism to regenerate the sinner with water at first,---the food of the communion table, made flesh and blood by the consecration of a priest, to sustain the life so begun, and the anointing with oil at last to prepare the soul for the burial. Such are the inevitable fruits of the sacramental theory, consistently carried out in the Church of Christ, making the very temple of God to be the habitation of every carnal and unclean thing.\footnote{Bellarm. \emph{Disputationes}, tom. iii. lib. ii. cap. i.-xxii. Perrone, \emph{Praelectiones Theologicae}, Parisiis 1842, tom. ii. pp.~5-66. Amesius, \emph{Bellarm. Enerv}. tom, iii. lib. i. cap. i.-v. Turrettin, \emph{Opera}, tom. iii. loc. xix. qu. i.-ix. Cunningham, \emph{Works}, vol.~iii. pp, 121-133. {[}Bruce, \emph{Sermons on the Sacraments}, Wodrow Soc. ed. Edin. 1843, pp.~11-33. Newman, \emph{Lectures on Justification}, pp.~316, etc.; Tract No.~90, 2d ed. p.~13. Wilberforce, \emph{Doctrine of the Holy Eucharist}, 3d ed. Lond. 1854, pp.~17-38, 97-130. Goode, \emph{Doctrine of the Church of England as to the Effects of Baptism in the case of Infants}, 2d ed. Lond. 1850, pp.~3-10. \emph{Vind. of the `Defence of the XXXIX.Art.'} etc., 2d ed. p.~38 f. \emph{Unpublished Letter of Martyr to Bullinger}, Lond. 1850, pp.~11-13. Martensen, \emph{Dogmatik}, 4te Aufl. Kiel 1858, pp.~361, 364. Matthes, \emph{Comparative Symbolik}, Leipzig 1854, pp.~492-510. Thomasius, \emph{Dogmatik}, 3ter Theil, 2te Abth. Erlangen 1861, pp.~113-135.{]}}

\hypertarget{the-sacrament-of-the-lords-supper}{%
\chapter{The Sacrament Of The Lord's Supper}\label{the-sacrament-of-the-lords-supper}}

\hypertarget{nature-of-the-ordinance}{%
\section{Nature Of The Ordinance}\label{nature-of-the-ordinance}}

CHRIST, as Head of His Church, has dealt out to it with a guarded hand merely outward and visible rites. In the provision which He has made for it there is enough in the way of outward and sensible ordinances for creatures made up of flesh as well as spirit to repose upon for the strengthening and confirmation of their faith; and yet not enough to convert their religion from a spiritual to a bodily service, and to transmute their faith into sight. There are but two ordinances, properly speaking, that link the Spirit with the flesh in the Christian Church; and lend the aid of a seen and sensible confirmation to an unseen and saving faith. There is one ordinance adapted to, and, it may be, specially designed for the case of infants, whose moral and intellectual life, still in the germ, lies hidden in a merely sensitive nature; and Baptism administered to the unconscious babe, whose ear cannot hear the word of salvation, becomes a visible and sensible token and seal impressed upon its flesh, of its interest in the covenant of its God. There is a second ordinance in a similar manner adapted for adults, in which an outward and sensible seal gives witness to their inward and unseen faith; and the Lord's Supper, preaching Christ by sign as well as word, is a fleshly witness, speaking to the flesh as well as to the spirit of the believer, of the blessings of the covenant of grace. There are these two, but no more than these two, outward and visible ordinances in the Church of Christ, like material buttresses, to strengthen and confirm a spiritual and immaterial faith,---the guarded and sparing acknowledgments of the fleshly nature, as well as the spiritual, which in the person of the Christian has shared in the sin, and shared also in the salvation from sin, which he knows.

We cannot doubt that a religion with these two, and neither more nor less than these two, outward rites is divinely proportioned and adapted to the need and benefit of our twofold nature, made up as it is of the fleshly and the spiritual, and both partners in the redemption, as they were formerly partners in the ruin, that belong to us. More than this in the way of the outward and sensible in the religion of Christ would have ministered all too strongly to the carnal and sensuous propensities of our nature, and would have tended towards a system which would have been ``meat and drink,'' and not ``righteousness, and peace, and joy in the Holy Ghost.'' Less than this in the way of outward and sensible ordinance would have left no room in the provision made in the Church for the adequate acknowledgment of our fleshly nature; and denied to our spiritual faith the benefit and support which it derives from some visible witness and confirmation of what it surely believes. Again, Baptism, as commonly administered to entrants into the Church, takes infeftment, so to speak, of our flesh when we enter into covenant with Christ, that not even the lower part of our being may be left without the attestation that He has redeemed it. The Lord's Supper, as administered from time to time to those who have been admitted into the Church before, renews this infeftment at intervals, and attests that the covenant by which we are Christ's still holds good both for the body and spirit which He has ransomed to Himself. The Sacrament of union to and the Sacrament of communion with Christ, tell that our very dust is precious in His sight, and has shared with the spirit in His glorious redemption. Other ordinances address themselves to the intellectual and moral nature exclusively, and speak of the care of Christ and the provision He has made for the growth and advancement of the spirit in all spiritual strength and life. The two ordinances of Baptism and the Lord's Supper, at different periods of our natural existence, and commonly in infancy and age, address themselves to both our outward and inward nature; and speak to us the testimony that both body and soul are cared for and redeemed by Christ, and that both in body and in soul we are His.

In formerly dealing with the case of Baptism as a sacramental ordinance, I endeavoured to ascertain its nature by an appeal to those marks or characteristics, in their application to Baptism, which we have found to define a Sacrament generally. Let us endeavour, by the same process, to make out the true nature and import of the Lord's Supper as a sacramental ordinance.

\hypertarget{the-first-mark-or-characteristic-of-a-sacrament-which-we-laid-down-is-that-it-be-a-divine-institute-appointed-by-christ-for-his-church.-there-is-no-religious-party-whatever-be-their-opinions-in-regard-to-the-meaning-of-the-ordinance-who-do-not-hold-the-divine-appointment-of-the-lords-supper-as-a-permanent-institution-in-the-christian-church-with-the-single-exception-of-the-quakers.-according-to-their-view-the-lords-supper-like-baptism-is-to-be-regarded-as-a-jewish-ordinance-and-the-practice-of-it-in-early-times-as-an-accommodation-to-jewish-prejudices-and-customs-but-an-ordinance-really-opposed-in-its-nature-to-the-spirituality-of-the-gospel-dispensation-and-not-intended-for-continuance-in-the-gospel-church.}{%
\subsection{The first mark or characteristic of a Sacrament which we laid down is, that it be a Divine institute appointed by Christ for His Church. There is no religious party, whatever be their opinions in regard to the meaning of the ordinance, who do not hold the Divine appointment of the Lord's Supper as a permanent institution in the Christian Church, with the single exception of the Quakers. According to their view, the Lord's Supper, like Baptism, is to be regarded as a Jewish ordinance, and the practice of it in early times as an accommodation to Jewish prejudices and customs, but an ordinance really opposed in its nature to the spirituality of the Gospel dispensation, and not intended for continuance in the Gospel Church.}\label{the-first-mark-or-characteristic-of-a-sacrament-which-we-laid-down-is-that-it-be-a-divine-institute-appointed-by-christ-for-his-church.-there-is-no-religious-party-whatever-be-their-opinions-in-regard-to-the-meaning-of-the-ordinance-who-do-not-hold-the-divine-appointment-of-the-lords-supper-as-a-permanent-institution-in-the-christian-church-with-the-single-exception-of-the-quakers.-according-to-their-view-the-lords-supper-like-baptism-is-to-be-regarded-as-a-jewish-ordinance-and-the-practice-of-it-in-early-times-as-an-accommodation-to-jewish-prejudices-and-customs-but-an-ordinance-really-opposed-in-its-nature-to-the-spirituality-of-the-gospel-dispensation-and-not-intended-for-continuance-in-the-gospel-church.}}

Now, in reference to this averment by the Quakers, it cannot be denied that, in the case of the Lord's Supper, as in the case of Baptism formerly noticed, our Lord adopted a Jewish practice or observance, and consecrated it as an ordinance in the Christian Church. The parts and ritual of the Supper are evidently derived from the observances connected with the passover as practised among the Jews. The Christian ordinance seems to be grafted upon the Jewish. We know from the Jewish accounts that we have of the passover service, that the master of the family or priest took unleavened bread, and broke it, and gave thanks to God, in much the same manner as we find it recorded of our Lord at the institution of the Supper. We know also from the same quarter, that there was one particular cup called ``the cup of blessing,'' or of ``thanksgiving,'' used at the paschal feast, of which the guests partook; and this was followed by the singing of psalms. These usages, connected with the Jewish passover, Christ adopted and accommodated to the ritual of that ordinance which we regard as the commemoration of His own death,---very much in the same manner as the washing with water employed in the Jewish baptisms or purifications was adopted and accommodated by Him to the other Sacrament which He established in the Christian Church. All this must be conceded to the Quaker theory in regard to the origin of the Christian Sacrament of the Supper. But all this, so far from making the ordinance a Jewish one, or justifying the explanation given by Quakers of the apos tolic practice of administering it, as a mere accommodation to Jewish customs or feelings, is very evidently calculated to demonstrate the reverse. The adoption of some parts of the paschal feast without the rest,---the eating bread and drinking wine as at the passover by Christians, without the slaying of the paschal lamb,---the observance of the practice at other times than once a year on the return of the anniversary of its first institution,---must, so far from being an accommodation or concession to Jewish feeling or prejudice on the part of the Apostles and first Christians, have been in reality a usage most repugnant to all the habits and prepossessions of the Israelites. The withdrawment of the outward ritual of the paschal service from the object of its original institution, and its destination to the purposes of a feast in commemoration of an event by which that service was abolished, were the very circumstances, above all others, calculated to make the ordinance not acceptable, but revolting, to Jewish feeling.

There is no truth, therefore, but the reverse, in the Quaker assumption, that the temporary continuance of the Lord's Supper in the Christian Church is to be accounted for on the theory of a concession to prejudices on the part of the Jewish converts. Add to this, that both in the statements of Scripture, and in the practice of apostolic men as recorded in Scripture, there is abundant evidence to prove that the Lord's Supper was no temporary ordinance, destined to pass away with the first merging of the Jewish into the Christian Church; but, on the contrary, was intended to be an abiding appointment for the use of its members. The command of our Lord to the disciples at the moment of the institution of the ordinance, spoke of its standing and permanent observance: ``This do in remembrance of me.'' The connection intimated by the Apostle Paul, in his account of the Supper, between the keeping of it and the second coming of Christ, evinces his opinion of the perpetual duration of the ordinance: ``As often as ye eat this bread, and drink this cup, ye do show forth the Lord's death till He come.'' The practice in the primitive Church, while under inspired direction in regard to the Lord's Supper, taken in connection with the absence of the faintest indication that it was meant for no more than a temporary purpose, is decisive evidence of the same conclusion. In short, the nature of the ordinance, as a memorial of Christ until that memorial shall be no more required on earth, in consequence of His second appearing,---the command to Jew and Gentile alike to keep the feast,---the universal practice of the Church under apostolic guidance,---and the absence of any statement express or implied in regard to the temporary character of the ordinance,---very clearly and abundantly demonstrate that the Supper of our Lord was a Divine and permanent appointment for the Church.

\hypertarget{the-next-mark-laid-down-by-us-as-characteristic-of-sacramental-ordinances-was-that-they-be-sensible-and-outward-signs-of-spiritual-truths-and-this-mark-applies-to-the-ordinance-of-the-lords-supper.}{%
\subsection{The next mark laid down by us as characteristic of sacramental ordinances, was, that they be sensible and outward signs of spiritual truths; and this mark applies to the ordinance of the Lord's Supper.}\label{the-next-mark-laid-down-by-us-as-characteristic-of-sacramental-ordinances-was-that-they-be-sensible-and-outward-signs-of-spiritual-truths-and-this-mark-applies-to-the-ordinance-of-the-lords-supper.}}

Simple and obvious although the idea be, that in the Lord's Supper we are commemorating, by appropriate and sensible images and actions, the grand spiritual truths characteristic of the Gospel, yet it is the omission or denial of this that has been the primary cause of numberless errors in regard to the nature of the ordinance. The Lord's Supper is not merely a commemoration; it is much more. But the fundamental idea which must be carried along with us in all our explanations of its nature and meaning is, that it is in the first instance a commemoration of the great truths connected with the death of Christ, as the sacrifice for the sins of His people. Nothing is easier, indeed, than to confound the sign with the thing signified; and nothing is more common in theological argument in reference to this matter. The nature and necessities of language lead us to attribute to the type what is only actually and literally true of the thing imaged or represented by the type; and in the frequent or common identification of the one with the other, we may be led not unnaturally to one or other extreme,---that of sinking the sign in the thing signified, or that of sinking the thing signified in the sign. The result is, either that we make the Sacrament to be nothing more than a sign, with no spiritual reality; or that we make it a mysterious spiritual reality, without being a sign at all. The identifying of the sign with the supernatural grace, and making them one and the same thing, must either lead to the Socinian notion that the Sacraments are nothing but symbols,---thereby evacuating the ordinance of all sacramental grace; or must lead to the Romanist or semi-Romanist notion that they are charms embodying and conveying spiritual grace, without regard to the spiritual meaning realized and appropriated by the believer in the ordinance. Hence the necessity and importance of bringing out distinctly, and laying down broadly, the character which Sacraments possess as signs of spiritual truths.

In regard to the Lord's Supper, nothing can be more distinct or conclusive than the commemorative character which is impressed upon the original institution of the ordinance by our Lord. With regard to the bread, the commandment was: ``Take, eat: this is my body broken for you: this do in remembrance of me.'' With regard to the second element in the ordinance---the cup---the appointment was no less explicit: ``This is the New Testament in my blood: this do ye, as oft as ye drink it, in remembrance of me.'' And in entire accordance with these declarations of our Lord as to the grand object of the Supper as commemorative, we have the further statement by the Apostle Paul, received by immediate revelation, as to the nature of the institution: ``For as often as ye eat this bread and drink this cup, ye do show forth the Lord's death till He come.'' In addition to all this, which very clearly exhibits the Sacrament of the Supper as in its first and most obvious character commemorative, we have the natural significance or pictorial meaning of the elements and actions in the ordinance. A rite may be in its sole or primary character commemorative in consequence of arbitrary appointment, although it may have nothing in itself naturally representative of the event commemorated. But this is not the case with the ordinance of the Communion Table. Over and above its positive institution in remembrance of the death and crucifixion of our Lord, there is a pictorial significance in the actions and elements of the Sacrament, fitted to keep constantly in view the grand and essential idea of the rite, as a rite of commemoration. The broken bread representing the broken and crucified body,---the wine poured out, the shed blood,---the eating and drinking of them, the participation in Christ's blessings to nourish the soul and make it glad,---the ``one bread'' and ``one cup,'' the communion of Christ with His people, and of them with each other,2---all these are no dumb or dark signs, but speaking and expressive of what it is intended to commemorate. This obvious characteristic of a sacramental ordinance, then, is most clearly seen in the Lord's Supper, that it is an outward and sensible sign of an inward and spiritual truth. It is the primary idea of the institution, never to be forgotten without infinite damage done to our understanding of its meaning, that, both naturally and by express Divine appointment, it is a symbolical and commemorative observance.

That the Sacrament of the Lord's Supper is an outward and sensible sign expressive of the grand and central truths connected with His death and sacrifice, is professedly held by all parties who hold that it is a Christian ordinance at all, and consider it to be binding upon Christians. And yet, notwithstanding of this professed and apparent unanimity upon the point, there is one religious denomination whose principles amount to a denial of this simple truth; and who virtually and really make the Lord's Supper to be not a sign, and not a commemorative ordinance at all,---thereby denying to it the proper character of a Sacrament. I allude to the Church of Rome. I do not mean to enter upon a consideration of the doctrine of that Church with regard to the Lord's Supper at present---for I intend to take up that subject afterwards,---but it may be not unsuitable or unimportant, meanwhile, to remark, that many of the errors of the Church of Rome in regard to this Sacrament are to be traced back to the neglect or denial of the simple but fundamental truth, that in its primary and essential character the Lord's Supper is a commemorative ordinance,---a remembrance of a sacrifice, and not a sacrifice itself,---a memorial of the great atonement and offering up of Christ on the Cross, and not a repetition of that atonement. By the doctrine of transubstantiation held by the Church of Rome, the elements of bread and wine are asserted to be changed into the actual body and blood of Christ, the Son of God; so that the use of these elements in the Sacrament is not to represent, but to repeat or continue the offering once made for sinners upon the Cross. The sign is identified with the thing signified; the symbol, instead of remaining a symbol, becomes one and the same with what was symbolized; the image and the reality are not two separate and independent things, but are confounded together. This is the unavoidable consequence of the doctrine of transubstantiation held in regard to the communion elements. The bread in the ordinance ceases to be the sensible sign of the Lord's body, and actually becomes that body; the wine in the cup ceases to be the representation symbolically of the blood of the Lord, and is transmuted into that very blood. There is no separating idea which continues to divide the symbol from the reality represented. The two are lost in one. The grand and fundamental characteristic of a Sacrament---that it is the outward and sensible sign of an inward and spiritual truth---is utterly forgotten or denied; and the consequence is the subversion of every idea essential to a Sacrament. While professedly, in some sort of way not easily understood, the Church of Rome holds that the Lord's Supper is a commemorative Sacrament, it in reality does away with the fundamental characteristic of a Sacrament as a sensible sign of spiritual truth.2

\hypertarget{the-third-mark-laid-down-by-us-as-characteristic-of-sacramental-ordinances-is-that-they-are-the-seals-of-a-federal-transaction-between-the-believer-and-christ-through-means-of-the-ordinance-and-this-mark-is-applicable-to-the-lords-supper.}{%
\subsection{The third mark laid down by us as characteristic of sacramental ordinances, is, that they are the seals of a federal transaction between the believer and Christ through means of the ordinance; and this mark is applicable to the Lord's Supper.}\label{the-third-mark-laid-down-by-us-as-characteristic-of-sacramental-ordinances-is-that-they-are-the-seals-of-a-federal-transaction-between-the-believer-and-christ-through-means-of-the-ordinance-and-this-mark-is-applicable-to-the-lords-supper.}}

There are not a few who rest contented with the position already laid down in regard to the Lord's Supper, and restrict themselves to the view which makes it a sensible sign of spiritual truth. At the date of the Reformation the subject of the Lord's Supper was very keenly canvassed amongst the Protestant Churches; and the Sacramentarian controversy, or the dispute as to the true meaning and nature of the Lord's Supper, went further than any other to divide the opinions of the early Reformers. While Luther held views approximating to those of the Church of Rome on this subject, although denying the doctrine of transubstantiation, there was another party among the first Reformers, especially in Switzerland, headed by Zwingli, who advocated principles differing very widely from those of Luther. Zwingli, the chief founder of the Protestant Churches in Switzerland, and the predecessor of Calvin in the Swiss Reformation, is not uncommonly regarded as the originator of those views of the Lord's Supper which represent it as a symbolical action commemorative of the death of Christ, and as nothing more than this. There seems to be good ground to question this opinion, and to doubt whether Zwingli ever really meant to deny that the Lord's Supper is a seal, as well as a sign of spiritual grace,---the outward voucher as well as representation of a spiritual and federal transaction between the believer and Christ through means of the ordinance. Under the strong reaction then felt from the views of the Lord's Supper entertained by the Church of Rome, which virtually set aside and denied the symbolical character of the ordinance, and superseded the outward sign by the thing signified, Zwingli and others felt that the true source of the doctrine of transubstantiation was the denial of the primary character of the ordinance as a commemorative sign, and the making the symbol give place to the reality symbolized under it. In other words, Zwingli and his associates in Switzerland held that the root of the evil lay in denying that the bread and wine in the Lord's Supper were signs, and constituting them the thing signified,---the very body and blood of the Lord. And in bringing out this principle as against the dogma of transubstantiation, they were led in their argument to speak somewhat unguardedly, as if, while Scripture represented the Sacrament as symbolical, it did not represent it as anything more than symbolical. Notwithstanding the violent controversy which the opinions of Zwingli and his followers excited, and the opposition they encountered from Luther and others of the German section of the Reformation, it is very doubtful indeed whether their opinion really excluded or denied the idea of a seal of a federal transaction, as well as a sign, as really belonging to the character of the Lord's Supper. However this may be, it was reserved for the successor of Zwingli, as the leader in the Swiss Reformation, to bring out from Scripture, and to establish on its true foundation, the proper notion of the Lord's Supper as more generally entertained by Protestant Churches since his time; and it is not the least of the many debts due by the Church to the illustrious Calvin, that we owe to him the first full and accurate development and decided maintenance of the true doctrine of the ordinance, as neither a sign alone, nor yet the thing signified alone,---as neither an empty symbol, nor yet the transubstantiated body and blood of Christ,---but as a sign and, at the same time, a seal of spiritual and covenant blessings, made over in the ordinance to the believer. The doctrine of the Sacrament of the Lord's Supper as a sign or symbol, and nothing more, has become the characteristic system of the Socinian party. More recently still, it has become the theory of not a few of the Independent body in England, as represented by Dr.~Halley.

That the Lord's Supper, in addition to being a sign, is also a seal of a federal transaction, in which the believer through the ordinance makes himself over to Christ, and Christ makes Himself over with His blessings to the believer, may be satisfactorily evinced from a brief review of the statements of Scripture on the subject. There are four different occasions on which the Lord's Supper is more especially referred to in Scripture; and from the statements made in regard to it on these occasions, it may be conclusively proved that much more is attributed to the ordinance than merely the character of a sign.

1st, There is the description given of the nature and meaning of the ordinance in connection with the history of its institution, as given by the different evangelists, and educed from a comparison of them, which seems not indistinctly to intimate that the Lord's Supper is more than a commemorative sign. In the words of the institution, our Lord calls the cup ``the New Testament or covenant in His blood,''---language which can be interpreted, and apparently requires to be interpreted, so as to assert a more intimate connection than any between a symbol and the thing signified, between the cup drunk in the Supper and the covenant of grace which secures the blessings represented. Add to this, that our Lord asserts the bread to be His body, and the wine to be His blood, in such terms as certainly imply that the one was a sign of the other, but apparently imply more than this,---the words seeming to intimate a sacredness in the symbols more than could belong to mere outward signs, and unavoidably suggesting a more intimate relationship between the elements of the ordinance and the spiritual blessings represented,---even such a connection as that which would make the use of the one by the worthy receiver stand connected with the actual enjoyment spiritually of the other.

2d, There is a separate account of the institution of the Lord's Supper given by the Apostle Paul in the 11th chapter of 1st Corinthians, in which the intimacy and sacredness of the connection between the symbols of the ordinance and the blessings represented are still more strongly brought out. The ``eating and drinking unworthily'' is represented as the sin of being ``guilty of the body and blood of the Lord;'' a second time it is spoken of by the apostle as the guilt on the part of the unworthy participator of ``eating and drinking judgment to himself,''---the reason assigned for the heinousness of the offence being, that he ``has not discerned the Lord's body;'' and, as a precaution against the danger of such transgression, a man is commanded to ``examine himself'' before he partake of the Supper. It seems impossible, with any show of reason, to assert that the ``discernment'' (διακρισις) here spoken of is the mere power of interpreting the signs as representative of Christ's death; or that the ``guilt'' incurred is nothing more than the danger of abusing certain outward symbols; or that the ``examination'' enjoined is no more than an inquiry into one's knowledge of the meaning of the commemorative rite. All these expressions evidently point to a spiritual discernment and participation by the believer, not of the sign, but of the blessing signified; and to a spiritual and awful sin, not of misusing and profaning outward symbols, but of misusing and profaning Christ actually present in them.

3d, There is a brief but most emphatic reference to the Lord's Supper in the 10th chapter of 1st Corinthians, which can be interpreted upon no principle which limits the meaning of the ordinance to a mere sign, but which very plainly asserts a federal transaction between the believer and Christ in the ordinance, and the communication through the ordinance of spiritual blessings. ``I speak as to wise men,'' says the apostle; ``judge ye what I say. The cup of blessing which we bless, is it not the communion of the blood of Christ? The bread which we break, is it not the communion of the body of Christ?'' The κοινωνια---the communion, or participation, or interchange, or mutual fellowship of the blood of Christ and the body of Christ---cannot possibly be understood of the mere signs of the body and blood, without a very violent experiment practised on the language of the apostle. And if ``the fellowship'' does not refer to the outward symbol, it can only refer to the spiritual blessings represented in the ordinance,---to Christ Himself present after a spiritual manner in the Sacrament, and giving Himself to the believer, while the believer gives himself to Christ, so as to establish a true κοινωνια, or fellowship, or communion between them. It is hardly possible with any plausibility to interpret the language of the apostle in any other way than as expressive of a federal transaction between the believer and Christ in the ordinance.

4th, There is a lengthened discourse in the 6th chapter of the Gospel by John, in which our Lord indeed makes no express reference to the Supper by name, but which it is hardly possible, I think, to avoid applying in its spiritual meaning to the ordinance. In that discourse our Saviour declares Himself to the Jews to be ``the bread of life which came down from heaven;'' He tells them that ``except they eat the flesh and drink the blood of the Son of man, they have no life in them;'' He asserts that ``His flesh is meat indeed, and His blood drink indeed;'' and He affirms that ``He that eateth my flesh, and drinketh my blood, dwelleth in me, and I in him.'' Whether this discourse refers directly and expressly to the ordinance of the Lord's Supper or not, it is quite plain that it affords, by the parallelism of the language employed to that used in connection with the ordinance, a key to interpret the sacramental phraseology applied to the Supper. It very plainly points to a spiritual eating and drinking of the flesh and blood of the Son of God, and a spiritual participation, far beyond a mere fellowship in an outward and empty symbol.4

On such grounds as these, we hold that the theory which explains the Sacrament of the Supper to be no more than a commemorative sign comes very far short of the Scripture representations of the ordinance; and that nothing but the idea of a seal of a federal transaction between the believer and Christ in the Sacrament will come up to the full import of the observance.

\hypertarget{the-fourth-and-last-mark-laid-down-by-us-as-characteristic-of-a-sacramental-ordinance-is-that-it-is-a-means-of-grace-and-this-mark-also-applies-to-the-ordinance-of-the-lords-supper.}{%
\subsection{The fourth and last mark laid down by us as characteristic of a sacramental ordinance, is, that it is a means of grace; and this mark also applies to the ordinance of the Lord's Supper.}\label{the-fourth-and-last-mark-laid-down-by-us-as-characteristic-of-a-sacramental-ordinance-is-that-it-is-a-means-of-grace-and-this-mark-also-applies-to-the-ordinance-of-the-lords-supper.}}

After what has been said, it is not necessary to do more than lay down this position. As the sign and seal of a federal transaction between the believer and Christ, it is plain that it must be the means of grace to his soul. It presupposes, indeed, the existence of saving grace on the part of the participator in the ordinance; it is a seal to him of the covenant actually and previously realized and appropriated by him; but, as a seal, it is fitted to add to the grace previously enjoyed, and to impart yet higher and further blessing. What is the manner in which this grace is imparted; how the Sacrament of the Supper becomes a living virtue in the heart of the participator; what is the efficacy of the ordinance,---these are questions the consideration of which opens up to us those further discussions to which we have next to address ourselves. While we believe that the Sacrament of the Supper is an eminent and effectual means of grace, as a seal of the covenant transaction represented in the ordinance, and through the faith of the participator, Romanists and semi-Romanists attribute to the ordinance a character and an efficacy which we believe that Scripture does not sanction, but, on the contrary, disowns. To the unscriptural views of the Supper held by the Church of Rome we shall now turn our attention.

\hypertarget{transubstantiation}{%
\section{Transubstantiation}\label{transubstantiation}}

Both the Lord's Supper and Baptism are Divine appointments of perpetual authority in the Christian Church. Both are outward and sensible signs, expressive of spiritual truths; both are seals of a federal transaction between Christ and the believer in the ordinance; and both, while they presuppose the existence of grace on the part of the receiver, are at the same time the means, by the Spirit, and through the believer's faith, of adding to that grace, and imparting a fresh spiritual blessing. And thus, parallel as the Sacraments of the Christian Church are in their nature and efficacy, they are alike also in the misapprehensions to which they have been exposed. Baptism has been misrepresented as an ordinance possessed in itself of an independent and supernatural virtue, apart from the spiritual state or disposition of the participator, so that, ex opere operato, it infallibly communicates saving grace to the soul. And, in like manner, the Sacrament of the Lord's Supper has been misrepresented as an ordinance embodying in itself a spiritual power, and efficacious of itself to impart saving grace. The full-grown and legitimate development of these views in regard to the Lord's Supper is to be found in the principles of the Church of Rome, and in the doctrine which she propounds under the name of transubstantiation.

The Romish system of belief and instruction in regard to the ordinance of the Supper is briefly this. At the original institution of the ordinance, it is believed by the Church of Rome that our Lord, by an exertion of His almighty power, changed miraculously the bread and wine into His body and blood, His human soul and His Divine Godhead; that this supernatural change was effected in connection with the words of institution uttered by Him: ``This is my body; this is my blood;'' that in giving the appearance of ordinary elements into the hands of His Apostles, He actually gave Himself, including both His humanity and His Divinity; and that they really received and ate His flesh, and drank His blood, with all their accompanying blessings to their souls. And what was thus done in a supernatural manner by Christ Himself at the first institution of the ordinance, is repeated in a manner no less supernatural every time the Lord's Supper is administered by a priest of Rome with a good intention. The priest stands in the place of Christ, with an office and power similar to Christ's, in every case in which he dispenses the Supper; the words of institution repeated by the lips of the priest are accompanied or followed by the same supernatural change as took place at first; the substance of the bread and wine used in the ordinance is annihilated, while the properties of bread and wine remain. In place of the substance of the natural elements, the substance of Christ in His human and Divine nature is truly present, although under all the outward attributes of bread and wine; and those who receive what the priest has thus miraculously transubstantiated are actual partakers of whole Christ, under the appearance of the ordinary sacramental elements.

Under this fearful and blasphemous system there are properly two grand and fundamental errors from which the rest flow; and which it is important to mark and deal with separately, although they are intimately connected, and form part of the same revolting theory of the Sacrament. There is, first of all, that supernatural change alleged to be wrought upon the elements by the authority of the priest in uttering the words of institution,---the transubstantiation properly so called,---by which the bread and wine become not a sign or symbol, but the actual substance of the crucified Saviour; and there is, secondly, and in consequence of such transubstantiation, the making of the elements not the signs of Christ's sacrifice, but the reality of it,---the bread and wine having become Christ Himself, and the priest having, in so transubstantiating them, actually made the sacrifice of the Cross once more, and offered it to God. These two doctrines of real transubstantiation, and a real sacrifice in the ordinance of the Supper, are both avowed as fundamental in the theory of the Church of Rome; and from these two doctrines all the others connected with the subject are derived. First, From the doctrine of the transubstantiation of the elements into the actual humanity and Divinity of the Lord Jesus Christ, there very obviously, and perhaps not unnaturally, follows that other doctrine, which declares that the elements are proper objects for the worship of Christians; and hence we have the elevation and adoration of the Host in connection with the Romanist doctrine of the Supper. Second, From the doctrine that the elements, transubstantiated into a crucified Saviour, become a real sacrifice, and a true repetition or continuation of the offering made upon the Cross, there very obviously and naturally follows that other doctrine, which teaches that the ordinance procures for the participator in it atonement and forgiveness of sin; and hence we have the saving grace infallibly communicated by the Sacrament wherever there is a priest to dispense it, or a soul to be saved by the participation of it. We shall consider, then, the doctrine of the Church of Rome in connection with the Supper, under the twofold aspect of the real transubstantiation alleged to pass upon the elements, and the real sacrifice alleged to be offered in the ordinance. These two points form the grand and essential features of the Romanist theory of this Sacrament; and, separately discussed, will enable us to review all that is of chief importance connected with it.

The doctrine of transubstantiation is thus laid down in the Canons of the Council of Trent: ``If any shall deny that in the Sacrament of the most holy Eucharist there is contained truly, really, and substantially the body and blood, together with the soul and Divinity of our Lord Jesus Christ, and so whole Christ, but shall say that He is only in it in sign, or figure, or virtue, let him be accursed.'' ``If any shall say that in the Holy Sacrament of the Eucharist there remains the substance of bread and wine, together with the body and blood of our Lord Jesus Christ, and shall deny that wonderful and singular conversion of the whole substance of the bread into the body, and of the whole substance of the wine into the blood, while only the appearances (species) of bread and wine remain---which conversion the Catholic Church most aptly styles transubstantiation,---let him be accursed.'' ``If any shall say that Christ, as exhibited in the Eucharist, is only spiritually eaten, and not also sacramentally and really, let him be accursed.''

This monstrous and audacious perversion of the doctrine of Scripture by the Church of Rome is founded upon and defended by an appeal to the literal meaning of the words of Scripture in speaking of the ordinance, in contradistinction to the figurative meaning of them. It is on this literal sense of the Scripture language that the only argument of Romanists in support of their system is built; and, over and above an appeal to the bare literalities of the expressions employed, there is not the shadow of a reason that can be alleged in defence of it. ``It is impossible for me,'' says Cardinal Wiseman in his Lectures on the Principal Doctrines and Practices of the Catholic Church,---``it is impossible for me, by any commentary or paraphrase that I can make, to render our Saviour's words more explicit, or reduce them to a form more completely expressing the Catholic doctrine than they do themselves: `This is my body; this is my blood.' The Catholic doctrine teaches that it was Christ's body, that it was Christ's blood. It would consequently appear as though all we had here to do were simply and exclusively to rest at once on these words, and leave to others to show reason why we should depart from the literal interpretation which we give them.'' Since Romanists, then, take up their position in defence of transubstantiation substantiation on the literal construction of the words employed in reference to the ordinance, and on that alone, what is material or essential to the argument is brought within a very narrow compass indeed. That argument may be, and indeed often is, encumbered with much irrelevant matter. But the main and only essential point to be discussed is simply this: Are we bound to interpret the Scripture phraseology employed in connection with the Lord's Supper in a literal sense, as affirming that the true body and blood of Christ are given in the ordinance; or, do the very terms of that phraseology, and the nature of the thing spoken of, compel us to adopt not a literal, but a figurative interpretation? This is evidently the status quœstionis between the Romanists and their adversaries in reference to the debate about transubstantiation. Romanists never pretend to bring any argument in aid of their theory of the Supper, except the argument of the literal meaning of the sacramental words. This disposed of, there is no other in the least available to defend their position. Is it, then, possible to adopt a literal interpretation of the words which Scripture employs to describe the sacramental elements? Is it competent to adopt a figurative interpretation? Is it necessary to adopt a figurative interpretation? These three questions, fairly answered, will embrace the whole controversy necessary to the discussion of the Romanist dogma of transubstantiation.

\hypertarget{it-is-impossible-to-adopt-a-literal-interpretation-of-the-sacramental-phraseology-and-this-is-evinced-by-romanists-themselves-in-their-own-departure-from-it-in-the-very-matter-under-discussion.}{%
\subsection{It is impossible to adopt a literal interpretation of the sacramental phraseology; and this is evinced by Romanists themselves, in their own departure from it in the very matter under discussion.}\label{it-is-impossible-to-adopt-a-literal-interpretation-of-the-sacramental-phraseology-and-this-is-evinced-by-romanists-themselves-in-their-own-departure-from-it-in-the-very-matter-under-discussion.}}

The principle of a strictly literal interpretation of the sacramental language of Scripture is the only principle which furnishes a single plea in favour of the dogma of transubstantiation; and yet the necessities of the language employed compel Romanists to surrender that principle in its application to the very case in which they demand that we shall observe it. The advocate of transubstantiation, by his own practice in the very matter in hand, nullifies his own solitary argument. He demands from us a literal rendering of the Scripture language; and yet in the very same passage of Scripture he is himself forced to adopt a non-literal. Take the words of Luke as he records the first institution of the Supper, and we see at once that in these the Romanist is forced again and again to abandon a literal, and have recourse to a figurative interpretation. ``And He took the cup,'' says the evangelist, describing our Lord's action, ``and gave thanks, and said, Take this, and divide it among yourselves.'' According to the strictly literal method of interpretation advocated and demanded by the Romanist, it was the cup, and not the wine in the cup, that was to be taken and shared by the disciples; and the Romanist is obliged to adopt the non-literal rendering in this case to suit his views of what occurred. Again, we find the inspired historian saying, in reference to what our Lord did, ``Likewise also the cup after supper, saying, This cup is the New Testament in my blood,''---language which once more demands that the Romanist shall surrender his literal, and have recourse to a non-literal interpretation, so that he may not identify the vessel in which the wine was contained with the New Covenant, nor transubstantiate the cup into a covenant, but make the one merely a sign or symbol of the other by a figurative use of the language. Once more, the Romanist departs from his principle of a literal interpretation, when the evangelist tells us that Christ spoke of His blood ``which is shed for you.'' At the moment of the utterance of these words, the shedding of His blood was a future event, to happen some hours afterwards, and not a present one, as the words literally rendered would assert; and, accordingly, the Romanist has no scruple in interpreting it in a non-literal sense, as indeed he is forced to do by the very necessity of the language. Or, take the words of the Apostle Paul in his account of the ordinance of the Supper, which he had, separately from the evangelists, himself received of the Lord. Here, again, we have the same use of terms which no literal interpretation will enable even the Romanist to explain. The apostle, like the evangelist, tells us that the words of our Lord were expressly, ``This cup is the New Testament in my blood,''---language which, interpreted upon the principle of strict literality, would identify the vessel containing the wine with the Divine covenant, and which requires, therefore, even in the opinion of the Romanist, to be understood figuratively. And, further still, the apostle, after the giving of thanks by our Lord, still speaks of the elements, not in language which denotes their transubstantiation, but in terms which plainly declare that they were bread and wine still. ``For as often as ye eat this bread, and drink this cup, ye do shew the Lord's death till He come.'' In this case no literal rendering of these words will be sufficient to reconcile them with the dogma of transubstantiation; and even in supporting that dogma, the Romanist is compelled in this passage to fall back upon an interpretation not literal. We are warranted, then, by the practice of Romanists themselves, in the very case of the sacramental language employed in Scripture, to say that it is not possible to adhere to, or consistently to carry out, a strictly literal interpretation.3

\hypertarget{a-figurative-interpretation-of-the-sacramental-language-is-perfectly-competent-and-possible.}{%
\subsection{A figurative interpretation of the sacramental language is perfectly competent and possible.}\label{a-figurative-interpretation-of-the-sacramental-language-is-perfectly-competent-and-possible.}}

It cannot be denied---and we have no occasion or wish to deny it---that, as a general canon of interpretation, it is true that the literal rendering of any statement made by a writer ought, in the first instance, to be tried and to be adopted, if it be in accordance with the use of words and the import and object of the statement. But the necessities and use of language justify and demand a figurative interpretation of terms, rather than a literal, in manifold instances; and those instances in which words are to be rendered not literally, but figuratively, must plainly be determined by the nature, connection, and object of the words. Now, in reference to the use of the sacramental language found in the Bible, it has often been argued, and has never yet been fairly met by the advocates of a literal meaning, that many similar passages are to be found in Scripture in which the same words admit of, and indeed require, not a literal, but a figurative interpretation, by the confession of all parties; and the conclusion is drawn from this, and fairly drawn, that the terms used in regard to the ordinance of the Supper may be figurative too. The occurrence of such texts, demanding, as all parties allow, a figurative or non-literal rendering, is valid and relevant evidence in regard to the nature of Scripture language, and proves at least this, that the words employed in reference to the Supper may admit of a figurative rendering also. This citation of parallel language does not in itself, indeed, demonstrate that the sacramental terms must be figurative; but it unquestionably proves that they may be figurative. Cardinal Wiseman, in his discussion of the doctrine of transubstantiation, gives a list of some texts bearing on the question, which have been referred to by Protestants as evidence in their favour, to the effect that the language, ``This is my body,'' ``this is my blood,'' may be understood, not literally, but figuratively. They are to the following effect:

``The seven good kine are seven years.''\\
``The ten horns are ten kings.''\\
``The field is the world.''\\
``And that rock was Christ.''\\
``For these are the two covenants.''\\
``The seven stars are the angels of the seven churches.''\\
``I am the door.''\\
``I am the true vine.''\\
``This is my covenant between me and you.''\\
``It is the Lord's passover.''

In these instances, and many similar ones, it is admitted by all parties, Romanists as well as Protestants, that the verb to be must be understood in its non-literal signification, and cannot by any possibility be understood literally. From the nature of the assertion made, from the context, and from the manner in which the terms are made use of, there is no possibility of denying that these texts are to be understood not literally, but figuratively; and they seem, therefore, by this parallelism to the words employed in connection with the Supper, to prove all that they were ever quoted to prove, namely, that the expressions, ``This is my body,'' ``this is my blood,'' may be understood in a figurative sense too. Such texts are not quoted to demonstrate that the sacramental phraseology of Scripture must be figurative; they are only quoted to prove that there is nothing in the nature of Scripture language, judging by its use in similar cases, to prevent us, if the nature of the statement and the context should require it, from interpreting the language concerning the Supper in a non-literal or figurative sense also. The multitude of texts closely analogous in form to the phrases, ``This is my body,'' ``this is my blood,'' and which, as all parties allow, must be understood figuratively, may not indeed, taken singly, necessitate a non-literal rendering in the latter case also; but they, at the very least, authorize it, should the import and connection of the passage make the demand, if they do not go a step further, and of themselves recommend a figurative interpretation.

Now, how is it that Cardinal Wiseman in his Lectures deals with these passages, and disposes of the argument drawn from them? He bestows a vast deal of minute criticism upon them, in order to show that these passages must, either from the meaning of the statement made in each, or the sense of the context, or the express assertion of the sacred writer, be accounted figurative and symbolical; and that, therefore, the verb to be in each of these cases must be reckoned equivalent to the verb to signify. And having done this, he considers he has done enough to prove that the cases referred to are not parallel to the sacramental language, ``This is my body,'' ``this is my blood.'' Now, it is enough, in reference to such an argument, to say that we willingly adopt his explanation of these passages, accounting them, as he does, to be figurative, and reckoning, as he does, the verb to be, when employed in such texts, as equivalent to the verb to signify. And it is for this very reason that we quote them as a justification of our assertion, that the same verb, when employed in reference to the Lord's Supper, may be equivalent there also to the verb to signify. If these texts did not admit of a figurative interpretation, and if the verb to be did not in them appear equivalent to the verb to signify, we should not have quoted them, because they would not have served our purpose. The reasoning of the Cardinal is certainly a singular specimen of an attempt at logical argument. I shall give it in his own words: ``Suppose,'' says he in his Lectures, ``suppose I wish to illustrate one of these passages by another, I should say this text, `The seven kine are seven years,' is parallel with `The field is the world,' and both of them with the phrase, `These are the two covenants;' and I can illustrate them by one another. And why? Because in every one of them the same thing exists; that is to say, in every one of these passages there is the interpretation of an allegorical teaching,---a vision in the one, a parable in the second, and an allegory in the third. I do not put them into one class because they all contain the verb to be, but because they all contain the same thing. They speak of something mystical and typical,---the interpretation of a dream, an allegory, and a parable. Therefore, having ascertained that in one of these the verb to be means to represent, I conclude that it has the same sense in the others; and I frame a general rule, that wherever such symbolical teaching occurs, these verbs are synonymous. When, therefore, you tell me that `this is my body' may mean `this represents my body,' because in those passages the same word occurs with this sense, I must, in like manner, ascertain not only that the word to be is common to the text, but that the same thing is to be found in it as in them; in other words, that in the forms of institution there was given the explanation of some symbol, such as the interpretation of a vision, a parable, or a prophecy \ldots{} Until you have done this, you have no right to consider them all as parallel, or to interpret it by them.''

The objection here urged by Cardinal Wiseman seems to amount to this, that we have quoted passages which, by the nature of the statement they contain, or by the context, or by the direct assertion of the writer, are plainly demonstrated to be figurative, while the sacramental expressions, ``This is my body,'' ``this is my blood,'' are not so demonstrated to be figurative. The answer is obvious. We do not quote such texts to prove that the terms of the sacramental institution must be understood figuratively, but to prove that they may be understood figuratively; to demonstrate that there is no bar in the shape of Scripture usage in the way to prevent us from interpreting them figuratively, if it is necessary. We are prepared to prove, by the very same means as the Cardinal employs,---by the nature of the statement itself, by the context, and such like considerations,---that the sacramental terms are figurative, just as Cardinal Wiseman proves that the words, ``This cup is the New Testament,'' are to be understood figuratively, or as these other terms, ``The seven kine are seven years,'' must be interpreted figuratively. The very nature of the statement itself proves it to be a statement to be understood, not in a literal, but a figurative sense. We interpret the expression, ``The seven kine are seven years,'' in a figurative sense, not because these words occur in the interpretation of a dream,---for both the dream and the interpretation may be embodied in words, literal, and not figurative,---but because the very nature of the proposition and the sense of the context necessitate it, it being impossible that the seven kine can be literally seven years. Again, we interpret, and so does Cardinal Wiseman, the expression, ``This cup is the New Testament,'' not literally, but figuratively, for a similar reason,---that the very nature of the proposition, and the sense of the context, demand a non-literal rendering; and in like manner we interpret the expression, ``This is my body,'' ``this is my blood,'' not literally, but figuratively, for the very same reason, because the very nature of the proposition, and the sense of the context, necessitate such an interpretation. The citation of other passages of Scripture in which the verb to be is used for the verb to represent or signify, is had recourse to in the argument simply to prove that the usage of Scripture language does not forbid, but countenances such a kind of interpretation. And the numerous texts already referred to are both relevant and sufficient to accomplish that object.

\hypertarget{a-figurative-interpretation-of-the-sacramental-language-this-is-my-body-this-is-my-blood-is-not-only-possible-and-competent-but-necessary.}{%
\subsection{A figurative interpretation of the sacramental language, ``This is my body,'' ``this is my blood,'' is not only possible and competent, but necessary.}\label{a-figurative-interpretation-of-the-sacramental-language-this-is-my-body-this-is-my-blood-is-not-only-possible-and-competent-but-necessary.}}

In no other way can we ever discriminate between figurative and literal terms, whether scriptural or non-scriptural, whether used by inspired or uninspired men, than by a reference to the nature of the proposition which the language embodies, to the sense of the context, and to the object of the speaker or writer; unless in those exceptional cases in which he directly tells us that he is to be understood in the one way or in the other. Very seldom indeed, in regard to language not meant to deceive, is it difficult to understand, from a consideration of these points, whether it is to be interpreted figuratively or not. In the case of the Lord's Supper, the words employed in reference to the elements could have presented to the disciples who heard them no difficulty at all. The ordinance was grafted upon the passover, with the figurative language and actions of which the Apostles, as Jews, were abundantly familiar; and this circumstance alone must have familiarized their minds with, and prepared them for the figurative meaning of the words and elements in the Supper. Above all, the nature of the proposition, ``This is my body,'' ``this is my blood,'' interpreted by the commentary of our Lord, ``This do in remembrance of me,'' and understood in the light of His accompanying actions and words, renders it nearly impossible that they could believe that a miracle had been wrought on the bread and wine, and that the body and blood, soul and Divinity of the Lord Jesus Christ, then present to their eyes, could be at the same instant contained under the appearance of the morsel of bread and the mouthful of wine that they ate and drank. Nothing but the ``strong delusion that believes a lie'' can lead any man who reads and understands the simple narrative of Scripture, to deny that the interpretation of the sacramental phraseology employed must be figurative, and not literal.

There are two attempts commonly made by Romanists to explain away the impossibility of the Apostles,---or indeed any other man not wholly blinded by spiritual delusion,---believing in the literal interpretation of the sacramental words that refer to the Supper.

1st, The power of Christ to work a miracle, like that which is alleged to have been wrought in the case of the bread and wine, is asserted; and it is averred that the Apostles could not doubt the supernatural ability of their Lord and Master, so often in other days exerted before their eyes. ``What,'' asks Dr.~Wiseman, ``is possible or impossible to God? What is contradictory to His power? Who shall venture to define it further than what may be the obvious, the first, and simplest principle of contradiction,---the existence and simultaneous non-existence of a thing? But who will pretend to say that any ordinary mind would be able to measure this perplexed subject, and to reason thus: `The Almighty may indeed, for instance, change water into wine, but He cannot change bread into a body?' Who that looks on these two propositions with the eye of an uneducated man, could say that in his mind there was a broad distinction between them, that while he saw one effected by the power of a Being believed by him to be omnipotent, he still held the other to be of a class so widely different as to venture to pronounce it absolutely impossible?.\ldots{} Now, such as I have described were the minds of the Apostles,---those of illiterate, uncultivated men. They had been accustomed to see Christ perform the most extraordinary works. They had seen Him walking on the water, His body consequently deprived for a time of the usual properties of matter,---of that gravity which, according to the laws of nature, should have caused it to sink. They had seen Him, by His simple word, command the elements and raise the dead to life, etc. Can we, then, believe that with such minds as these, and with such evidences, the Apostles were likely to have words addressed to them by our Saviour, which they were to interpret rightly, only by the reasoning of our opponents,---that is, on the ground of what He asserted being philosophically impossible?''

It is hardly necessary to reply to such an argument as this. In the first place, the miracles with which the Apostles were familiar had no analogy whatsoever to the stupendous wonder of transubstantiation. Those miracles were appeals to the senses in proof of truths not seen; and they were tested by the senses, as things to be judged of by them all. The so-called miracle of transubstantiation is no appeal to the senses, but the reverse,---a thing not to be tested by the exercise of any one of them, if it were possible, and a thing denied by any one of them, because impossible. If it were a possible thing, it would subvert the very principle on which our perceptions are made to us by God the primary source of our beliefs, and the foundation of truth to us; and it would cause the very instincts which His hand has laid deep within our inmost being to be to us a lie. The conversion of water into wine at that marriage supper in Cana of Galilee of old was a wonder seen by the eye, and in agreement with the evidence of the senses, because the properties, first of the water, and afterwards of the wine, were seen and judged of by all. The conversion of the bread into the body of the Lord, while yet the properties of bread remain, is a wonder that contradicts the evidence of our senses, and involves an impossibility.

In the second place, even Cardinal Wiseman himself admits that there are impossibilities in the nature of things, not competent even for Almighty power to accomplish. Such an impossibility, according to his own statement, is the ``existence and simultaneous non-existence of a thing;'' and side by side with this one limitation, which, upon the authority of Dr.~Wiseman, is to be put even upon the power of God, we may put another limitation, and that upon higher authority than his: ``God cannot deny Himself.'' In that revelation which He has given to us in our instinctive and primary perceptions of sensible things, and in that other revelation which He has given to us in His Word, God, who is the Truth, cannot contradict Himself.3

2d, An attempt is made by Romanists to identify, as one and the same in principle, the dogma of transubstantiation and what are called the mysteries of revelation. ``What,'' says Cardinal Wiseman, ``becomes of the Trinity? What becomes of the incarnation of our Saviour? What of His birth from a virgin? And, in short, what of every mystery of the Christian religion?'' It will be time enough to answer such questions as these when it is proved that such mysteries contradict our rational nature, in the same manner as the dogma of transubstantiation contradicts our perceptive nature. Such mysteries as those referred to are above our reason, but not against it. They are beyond the powers of our rational nature fully to understand, but not contradictory to our rational nature so as to be inconsistent with it. The argument in defence of transubstantiation, drawn from such a source, is but one example out of many that could be quoted, of the common tactics of Romish controversialists, who are but too often prepared to hand over to the unbeliever the most sacred truths which the Scripture has recorded, rather than not make out a plea for their own superstitions.

\hypertarget{the-doctrine-of-the-real-presence-and-the-priestly-theory}{%
\section{The Doctrine Of The ``Real Presence'' And The Priestly Theory}\label{the-doctrine-of-the-real-presence-and-the-priestly-theory}}

With the dogma of transubstantiation, as held by the Church of Rome, stands very closely connected the question as to the manner in which Christ is present in the ordinance of the Supper. The doctrine of the ``real presence'' of Christ in the Sacrament has, more almost than any other in theology, been made the subject of prolonged and bitter controversy. By the Church of Rome, as we have seen, the real presence of Christ is explained to be the true and actual existence of the body and blood, the soul and Divinity of the Saviour, under the sensible appearances of bread and wine; so that in the elements Christ is as much present after a bodily sort, in consequence of their transubstantiation, as He ever was present to His disciples of old in the days of His flesh. By the Lutheran Church, the real presence of Christ in the ordinance is maintained, not upon the principle of such a change in the substance of the elements into Christ's body and blood as contradicts the testimony of our senses, but, rather upon the supposition that the bread and wine remaining the same, the real body and blood of Christ are nevertheless united to them in some mysterious manner, so as to be actually present with them, and actually received along with them, when they are partaken of by the communicant. By our own Church, as well as by many other Protestant communions, the real presence of Christ in the Sacrament is asserted on the ground that He is not in a bodily manner present in the substance of the elements, nor yet in a bodily manner mysteriously present with the elements, but only spiritually present to the faith of him who receives the ordinance in faith.

The influence of the fierce and frequent controversies waged in connection with the nature and efficacy of the Lord's Supper shortly after the date of the Reformation, and the disposition on the part of Luther, and the Churches affected by his influence, to depart as little as possible from the established phraseology of the ancient Church on the subject of the Sacrament, served to introduce, or to continue in theological discussions, a language somewhat exaggerated, and occasionally almost unintelligible, in regard to this question. Such, undoubtedly, was the phrase ``consubstantiation,'' used by some of the Lutherans to express the mysterious corporeal presence of Christ, not in, but with, or under, or somehow in connection with the elements; and such also was the phrase ``impanation,'' employed by others to elucidate, or rather to obscure, the doctrine of the manner in which Christ's bodily presence is connected with the sacramental bread. And I cannot help thinking that, under the power of very much the same influences, the term ``real presence'' has not unfrequently been employed and explained, even by orthodox divines, in such a way as to give a somewhat exaggerated and mysterious aspect to the connection subsisting between Christ and the Sacrament. That phrase has occasionally been employed in association with such language as to leave the impression that Christ was present in the Supper, not spiritually to the faith of the believer, and not corporeally to the senses of the communicant, but in some indefinite manner between the two, and after a sort mysterious and peculiar to the Sacrament of the Supper. Such language seems to have no warrant in the Word of God.

The Scriptures give us no ground to assert that Christ is present in the Sacrament of the Supper in a manner different from that in which He is present in the Sacrament of Baptism. I do not speak at present of the extent of the blessing or of the grace which He may impart in the one or the other Sacrament by His presence; I speak only of the manner of His presence. There is nothing, I think, in Scripture to warrant us in affirming that the manner of Christ's presence in the Supper is in itself unique or peculiar, or indeed in any respect different from the manner of His presence in Baptism, or any other of His own ordinances. In all of these He is present, after a spiritual manner, to the faith of the participator in the ordinance, and in no other way. The blessings which that presence may impart may be different in different ordinances, and may be more or less in one than in another. But there is nothing in the Word of God which would lead us to say that the real presence of Christ in any of His ordinances, whether sacramental or not, is anything else than Christ present, through his Spirit and power, to the faith of the believer. Such promises as these---``Lo, I am with you alway, even unto the end of the world;'' ``Where two or three are met together in my name, there am I in the midst of you;'' ``Behold, I stand at the door and knock: if any man hear my voice, and open the door, I will come in to him, and will sup with him, and he with me;'' and such like---plainly give us ground to affirm that Christ, through His Spirit, is present in His ordinances to the faith of the believer, imparting spiritual blessing and grace. But there is nothing that would lead us to make a difference or distinction between the presence of Christ in the Supper and the presence of Christ in His other ordinances, in so far as the manner of that presence is concerned. The efficacy of the Saviour's presence may be different in the way of imparting more or less of saving grace, according to the nature of the ordinance, and the degree of the believer's faith. But the manner of that presence is the same, being realized through the Spirit of Christ, and to the faith of the believer. The Sacramentarian controversy has tended in no small measure to introduce into the language of theology, in connection with the ``real presence,'' an ambiguity of thought and statement, not confined to Romanist, or even semi-Romanist divines.

But, passing from that part of the Popish theory of the Supper which refers to the alleged change produced on the elements by transubstantiation, and to the manner of Christ's presence in the ordinance, I go on to consider the other part of the Popish theory of the Supper which refers to the office of the ministering priest in the Sacrament, or his power to offer the body and blood of Christ, actually present, as a true sacrifice for sin. The first grand error in the Popish doctrine of the Lord's Supper is the monstrous figment of the transubstantiation of the elements; the second, intimately connected with the first, and perhaps yet more extensive and mischievous as an error in its practical bearings, is the doctrine of the power of the Church, in the ordinance of bread and wine, to offer a true and efficacious propitiation to God, both for the living and the dead. The sacrifice of the mass is founded upon, and very closely connected with, the dogma of transubstantiation,---in some sort following as an inference from the assumption that the priest stands in Christ's stead at the Communion Table, and, by a supernatural power not inferior to Christ's, changes, by the utterance of the words of institution, the elements of bread and wine into the actual body and blood, soul and Divinity, which were once the sacrifice offered up for this world upon the Cross. In the performance of this supernatural and mysterious office, which, according to its own theory, it is given to the Church of Rome to discharge, we see both the priest and the sacrifice,---the priest, acting as mediator between God and the people, offering a true satisfaction to God for sin, and promising remission and reconciliation; and the sacrifice presented to God, real and efficacious, because in fact the very same sacrifice, in its substance, of the flesh and blood of Christ, as He Himself once made and presented, and not less availing in its mighty virtue to propitiate God, and procure salvation for the sinner. A real office of priesthood, and a real offering of sacrifice, are the two features that characterize this second portion of the Popish theory of the Sacraments. Both are asserted, and both are essential in the sacrifice of the mass, which has been grafted on the dogma of transubstantiation, and both form integral parts of that monstrous system of sacerdotal usurpation by which the Church of Rome seeks to build up her spiritual tyranny. The position, then, laid down by the Church of Rome in connection with the subject of the mass, may be conveniently discussed under these two heads: first, the claim which she makes to possess and exercise the office of a true priesthood; and second, the power that she arrogates to make and offer a true sacrifice to God. Reserving the second of these points for future consideration, we shall now proceed to deal with the claim put forth by the Church of Rome to hold and exercise the office of a real priesthood.

This claim runs through the doctrine and practice of the Popish Church in all its departments, and is not restricted to the case of its views in connection with the Supper. The priestly office and sacerdotal pretensions are recognised in almost every branch of its administration as a Church, and, indeed, are fundamental to the system. But the priesthood which it pretends to exercise towards God and on behalf of man is perhaps developed most prominently and conspicuously in connection with its doctrine of the Lord's Supper. The question is one that lies at the very root of the difference between the Popish and Protestant systems, and on that account is of more than ordinary interest and importance.

The doctrine of a real priesthood residing in the Christian ministry, more especially in connection with its chief function of offering the sacrifice of the mass, is thus stated by the Council of Trent: ``Sacrifice and priesthood are so joined together by the ordinance of God that they existed under every dispensation. Since, therefore, under the New Testament the Catholic Church has received the holy visible sacrifice of the Eucharist by the institution of the Lord, it is necessary also to confess that there is in it a new, visible, and outward priesthood into which the old has been transferred. Now the sacred writings show, and the tradition of the Catholic Church has always taught, that this was instituted by the same Lord our Saviour, and that a power was given to the Apostles, and their successors in the priesthood, of consecrating, offering, and administering His body and blood, and also of remitting and retaining sins.'' ``If any shall say that by these words, `Do this in remembrance of me,' Christ did not appoint the Apostles to be priests, or did not ordain that they and other priests should offer His body and blood, let him be accursed.'' ``If any shall say that the sacrifice of the mass is only one of praise and thanksgiving, or a bare commemoration of the sacrifice accomplished upon the Cross, but not propitiatory; or that it only profits him who receives it, and ought not to be offered for the living and dead, for sins, pains, satisfactions, and other necessities,---let him be accursed.''

Amid the other errors contained in these statements by the Council of Trent, what we have chiefly to do with at present is the claim which is put forth on behalf of the Church of Rome and her ministers to hold and exercise the office of priesthood in the same sense as, ceremonially, the priests of a former dispensation did so; with power now, not ceremonially, but really, to act as priests in the absence of Christ in heaven, and truly to offer sacrifice to God for sin. The question in regard to such a claim is this: Have we any warrant to believe that a visible and external priesthood has been established in the New Testament Church, with powers to act as mediators between God and man, and offer the propitiatory sacrifice for the living and the dead; or has the office of priesthood which existed under a former economy no longer an existence now in the Gospel Church, there being none on earth authorized or qualified to undertake it,---the one Priesthood, in the end of the world for sin, having completed its work on earth, and the Priest who held the office having returned to heaven to continue it there? This is a vital and fundamental question, not only in order to enable us to form an estimate of the real character of the system of Romanists, but also because it enters so essentially into the principles held by High Churchmen of other denominations.

\hypertarget{the-existence-of-a-priesthood-as-a-standing-ordinance-in-the-christian-church-is-inconsistent-with-the-fact-that-such-an-office-was-abrogated-with-the-jewish-economy-and-necessarily-came-to-an-end-when-that-dispensation-gave-place-to-the-gospel-economy.}{%
\subsection{The existence of a priesthood as a standing ordinance in the Christian Church is inconsistent with the fact that such an office was abrogated with the Jewish economy, and necessarily came to an end when that dispensation gave place to the Gospel economy.}\label{the-existence-of-a-priesthood-as-a-standing-ordinance-in-the-christian-church-is-inconsistent-with-the-fact-that-such-an-office-was-abrogated-with-the-jewish-economy-and-necessarily-came-to-an-end-when-that-dispensation-gave-place-to-the-gospel-economy.}}

An earthly priesthood was an ordinance appointed for a special purpose and a special time; and the purpose having been served, and the time past, it is necessarily at an end. The priestly office, and the institution of sacrifice with which it stands inseparably connected, formed part of that instrumentality by which, for thousands of years, God prepared this world for the coming and the death of His own Son as its Saviour. First of all, it was the father of the family who was ordained the priest to offer the sacrifice for the rest, and to approach unto God on behalf of his household; the members of which drew near to God, and worshipped, and were accepted only through him. Such seems to have been the practice in patriarchal times, and apparently not without the appointment, or at least the sanction, of God. The father of the family, as well as the divinely appointed sacrifice he offered, thus in a general and distant way represented Christ as the medium whereby sinners might approach to God in worship. But the patriarchal institute was too general and vague a type of the One Mediator through whom alone, when fully revealed, men were to find access to God. Accordingly it was done away with, and another institute was ordained in its place, with priests specially set apart to the office of mediators between God and the people, and with more special authority given, and more distinct provision made for them to be the media through whom the rest were to present their worship and sacrifices, and themselves to make their approach to God and find acceptance. Under the Mosaic ritual, it was no longer lawful for the sinner himself directly to approach to God with his own offering of worship or sacrifice; it was no longer lawful for the sinner even to draw near with his sacrifice unto God through the head of the family, as under the patriarchal institute. The avenue of approach to God was, step by step, narrowed and restricted. First, the father of the family was marked out and selected as the recognised priest and mediator for the rest. Next, a further limitation took place, and the priest of Aaron's line was specially appointed to stand in the stead of the whole families of the nation in their approach to God; and strict provision was made---and guarded by the most solemn penalties---that no man should venture to present the sacrifice himself, or to worship except through the media of this one commissioned priesthood. The thousands of Israel were restricted in their legal worship to the one avenue, and forbidden to draw near to the Holy One of Israel except through the one mediation of the earthly priest of Aaron's lineage.

And why was it that this earthly priesthood was thus marked off from all the rest, and the other worshippers made dependent on the one appointed priest of Aaron's house? And why were men forbidden to approach to God directly and immediately themselves, or even indirectly through any other but this one mediator? The answer is obvious. The priesthood was so restricted, and so fenced about with solemn limitations, in order that it might be a type of Christ, ``the one Mediator between God and man.'' From age to age, and from step to step, the worshippers of God under the old economies were more and more shut up to the idea and the practice of approaching the Most High God only through the channel of one Priesthood and the person of one High Priest. The typical priests and priesthoods of former dispensations led men's hearts and habits to fix upon the one Mediator through whom alone we now draw near to God. They taught the worshippers to anticipate and to hope in that one Man, who is now the Priest, not of one family, as in patriarchal times, nor of one nation, as in Jewish times, but the Priest through whom all the families and all the nations of the world draw night to God. The earthly priesthoods of the former days of the Church all converged upon and pointed to and centred in Christ. With Christ, therefore, those priesthoods came to an end. The type was merged in the Antitype, and then was done away. The priests of other days, together with the sacrifices which they offered, have served the object designed by them, and are abolished. They can, from the very nature of their office, have no use, and no meaning, and no place in a Church to which another and a higher priesthood has been given, and when the sign has given place to the thing that was signified. The office of the priesthood on earth ceased with the former dispensation; and not only is there no re-appointment under the Gospel of such an order of men in the Church, but they would, from the very place and office that they occupied, be inconsistent with the Gospel economy. They formed part and parcel of a typical system which has been abolished.

\hypertarget{the-existence-of-a-priesthood-as-a-standing-ordinance-in-the-christian-church-is-inconsistent-with-the-privileges-of-believers-under-the-gospel.}{%
\subsection{The existence of a priesthood as a standing ordinance in the Christian Church is inconsistent with the privileges of believers under the Gospel.}\label{the-existence-of-a-priesthood-as-a-standing-ordinance-in-the-christian-church-is-inconsistent-with-the-privileges-of-believers-under-the-gospel.}}

It is not unfrequently argued by the advocates of Romanist or semi-Romanist principles on this subject, that the privilege of a human priesthood and a human mediatorship is one so great and precious that it cannot be conceived to exist, as we know it did, under the earlier and far inferior dispensation, and yet to be awanting under the later and far better dispensation of the Gospel. The presence of an earthly priesthood, it is urged, must be enjoyed by the Church now, inasmuch as it cannot be supposed to be deprived of one of the highest privileges which belonged to the former and less richly endowed Church of the Old Testament.

A comparison between the superior advantages of the Gospel Church, as measured by those of the Jewish, is the very consideration which, instead of proving that a human priesthood is continued to us now, most emphatically demonstrates that it is abrogated. The presence and office of a human priesthood, enjoyed by worshippers under the law, are far surpassed by the higher and more glorious privileges enjoyed by believers under the Gospel. No doubt it was an act of grace and condescension on the part of God, to permit sinners to approach His presence through the avenue of a visible priesthood and a visible sacrifice in former times, even although that boon was granted to them under solemn and jealous restrictions; and it was a great and precious privilege for the worshipper to be allowed to draw near to the mercy-seat through means of a human mediator, and by the intervention of a material offering. But the privilege of Christians in the New Testament Church is better and more glorious still. Through Christ a new and living way has been opened up for all to draw nigh to God, not indirectly through a human mediator, but directly, each man for himself. The whole brotherhood of believers are no longer dependent upon one of themselves for the liberty or opportunity of access to the common Father; and without distinction of special office, it is the freedom purchased for all, without earthly priest or earthly intercessor interposed, to go with boldness into the very holiest. The presence of an earthly and external priesthood is no evidence of superior privilege, but the reverse. It is the mark of an imperfect and carnal dispensation.

That it was necessary for the worshipper to employ the intervention of another than himself in order that he might approach to his Creator,---that a sinner should be dependent on another sinner for pardon or access to heaven,---that he should not dare to engage his heart to draw near to God except through the medium of a human priesthood,---were strong arguments to prove the essential imperfection of that dispensation which witnessed such things, and constituted a yoke of bondage which it was hard to bear. And what it was when the sons of Aaron by God's own appointment were the human priests and mediators, that it is now in the case of those Churches who bind upon their own necks the institute of a human priesthood, and then boast of it as their exclusive distinction and privilege. It is a spiritual yoke that is too heavy to bear; it is a retrogression from the freedom wherewith under the Gospel Christ has made His people free; it is a badge of the voluntary thraldom and debasement of a Church that has itself gone into bondage to men, instead of maintaining the liberty of Christ the Lord. The restriction of approaching God only through the earthly priest in the local temple at Jerusalem, and by the blood of bulls and goats,---the prohibition forbidding the sinner to draw near to the mercy-seat directly himself, or through any other medium,---those were evidences of essential imperfection in the Church state of the worshippers under a former economy. And the human priesthood of the Church of Rome,---the material sacrifice made and offered for the worshippers,---the priest standing between the sinner and God, and barring or opening the way of approach,---the mediator acting as the medium of communication between the Most High and His creatures, and retaining or remitting their sin,---these, too, are restrictions, and, because human and unauthorized, daring and impious restrictions, upon the freeness of God's grace and the liberties of His redeemed people.

It is a fact of much significance, and indeed of decisive force in this argument, that throughout the whole of the New Testament Scriptures there is no instance in which either the name of priest, or the functions belonging to the office of priesthood, are ascribed to the ministers of the Christian Church; that the only examples of the use of the term are those in which it is given, not to the minister, but to the people; and that the ascription of the privileges of the office is uniformly made to the members at large. On the one hand, the term ἱερευς, or ``priest,'' is never in any single instance in the New Testament applied to a minister of the Christian Church, although always made use of to designate the priest of the Aaronic dispensation. The usual name given to the minister of the New Testament Church is πρεσβυτερος,---the change of designation marking very decisively the change in the nature of the office. On the other hand, on the only occasions on which the word ἱερευς is used in the New Testament in reference to any except a Jewish priest, it is given to the members of the Christian Church at large, and not to the ministers of that Church. In the Book of Revelation, believers are spoken of as ``kings and priests to God;'' and in the first Epistle of Peter they are described as a ``royal priesthood.'' The name formerly appropriated to the sons of Aaron, selected and anointed from among the rest of the congregation to be priests to God, is not inherited by the ministers of the Christian Church in the same exclusive manner, but, on the contrary, is applied in an enlarged and extended sense to the whole body of believers. More than this: the privilege enjoyed by the priests of old, of alone of all the worshipping assembly drawing near to God without the intervention of any other, is a privilege uniformly represented in the New Testament as not peculiar to the ministers of the Church, but extended now to all its members, and common to all believers. The office peculiar to the minister of the Christian Church is described at large in the New Testament Scriptures, and is a ``ministry'' or ``service'' unto others (διακονια, λειτουργια), not a mediatorship on behalf of others. It is spoken of as an office of ``ministering,'' ``preaching,'' ``exhorting,'' ``ruling,'' amid the flock of Christ, not an office of sacrificing, and making reconciliation, and approaching to God as the mediator on behalf of the rest, and becoming the avenue for the access of their persons or worship to the Divine presence. On the contrary, this privilege of approaching directly to God without the intervention of any substitute or proxy on earth, is a privilege which is expressly attributed to all believers as their personal right: so that, if in any sense there are priests now on earth, those priests are the believing people of God at large; and if in any sense there are priestly sacrifices now offered up, they are the spiritual sacrifices of the prayer and praise of Christians, without distinction of office or place in the Church. The sacerdotal theory on which the Church system of Rome is built, and the priestly office which is so conspicuously developed in her practice as regards the Lord's Supper, are utterly repugnant to the spirit of the New Testament Church, and to the privileges which it has secured to believers. The privilege of a human priesthood, which existed under the law, is abolished under the Gospel; or rather, in its spirit and substance, the privilege is enlarged and extended to all believers under the New Testament Church. It was the peculiar and distinctive prerogative of the priests under the law, that they alone of all the worshippers drew near to God without a human mediator. That prerogative is common to all the royal priesthood of believers under the Gospel.

\hypertarget{the-existence-of-an-earthly-priesthood-as-a-standing-ordinance-of-the-christian-church-is-inconsistent-with-the-one-office-of-christ-as-the-priest-and-mediator-of-his-people.}{%
\subsection{The existence of an earthly priesthood as a standing ordinance of the Christian Church is inconsistent with the one office of Christ as the Priest and Mediator of His people.}\label{the-existence-of-an-earthly-priesthood-as-a-standing-ordinance-of-the-christian-church-is-inconsistent-with-the-one-office-of-christ-as-the-priest-and-mediator-of-his-people.}}

Earthly priest the New Testament Church has none. The very name is blotted out from the inspired history of the Church under the Gospel in its application to any office-bearer within its pale; and it is found, in so far as it can now be found on earth, only in connection with that spiritual and universal priesthood which belongs alike to all true believers, who have equally the privilege of free approach to God, equally the anointing which makes them His people, and equally the consecration that sets them apart for His service. In any other sense than this, there is no priest in the Christian Church on earth. The material sacrifice made by men has ceased, the incense kindled by men no longer burns, the atonement presented by men is no more offered up. The Gospel is a religion without a priest on earth, without a sacrifice, and without an altar. And yet there is a priesthood that belongs to the Christian Church still; and there is a Priest who yet discharges that office on behalf of His people. ``We have a great High Priest that hath passed into the heavens for us,''---not a mortal and dying man, but one ``of whom it is witnessed that He liveth for ever,''---not a priest who offers, as did the sons of Aaron of old, the typical sacrifices of blood, or, as the ministers of Rome do now, the pretended sacrifices of an unbloody offering of bread and wine,---but one who, once for all, offered up a Divine yet human sacrifice for men,---not an intercessor, who, like the high priest under the law, entered into God's presence with the blood of bulls and goats, nor yet like the priest of the Papacy with a consecrated wafer,---but an Intercessor, who, with His own precious and more than mortal blood, has passed into the presence of God,---an Intercessor, the Son of God, presenting the offering of Himself without spot or blemish, and pleading for us on the ground of His meritorious sacrifice. And this office which the Son of God now discharges in heaven for His Church passes not from Him to any other (ἀπαραβατον ἐχει την ἱερωσυνην.) His is an unchangeable and undying Priesthood; and He ever liveth to make intercession for His people. The office which He sustains and discharges in heaven is His own incommunicable office, which none save Himself has either the right or the power to discharge. The one Priest that has made the sacrifice and offered it to God for the sins of many,---there was none that could share with Him in that mighty and mysterious work. The one Priest to stand between God and a sinful world,---there was none but the Son that could undertake so to approach unto the Most High. The one Priest to intercede with an offended God for the guilty,---there was none but the equal of the Father that could so plead. The one Priest to dispense unto men throughout all ages the blessings of redemption and grace,---there is none equal to the task but He ``in whom dwelleth all the fulness of the Godhead bodily.'' Alone in His office as in His nature, unapproachable in His work as in His greatness, ``He abideth a Priest for ever,''---the ever-present and ever-living Mediator, who has no fellow to share in His priestly functions, and whose glory as Mediator He will not give unto another.

And what shall we say of those Church systems, Romanist and semi-Romanist, that give to mortal men that office of Priest which none can bear but the Son of God, and constitute sinners mediators on earth between their fellow-sinners and the Almighty? Such an encroachment upon His incommunicable office touches very nearly the honour of Christ. The assumption by men of His personal and inalienable prerogatives, inseparable from Himself as Mediator, is a dishonour done to Him in that very character in which He stands forth supreme and alone before the eyes of the universe. The very title of Mediator belongs in the Christian Church to none but One, and He the only-begotten Son of the Father. Our lips are now forbidden to name another Priest but Jesus. Even in the Old Testament Church, the name and the office of the Priest had something in them of awful and mysterious import, typical as they were of the fulness of the Gospel day, and of the greatness of the Gospel Mediator, and fenced about, as we know them to have been, with the solemn and irrevocable sentence of death upon those who should unwarrantably assume or encroach upon them. And still more awful are that name and office of Priest, now that in these latter days they have been sustained by the Son of God Incarnate, and mysteriously sanctified by the shedding of that more than mortal blood which was poured out on Calvary, and which He still day by day presents in heaven, as He continually pleads with the Father there. To stand between God and man, as Christ once stood amid the darkness of Calvary, was a work which none but He could do. To stand between God and man, as Christ now stands, a Priest in heaven no less than on earth, is a work which none but He can accomplish. To bear the burden of such an office now is as little competent to mortal man as it was to bear the burden of it in the Garden, or at the Cross. The name of Priest between God and man is Christ's inalienable and incommunicable name,---whether He bears the anger of an offended Judge, or pleads with the compassion of a reconciled Father,---whether He makes, as He once did, atonement by sacrifice, or makes, as He now does, intercession by prayer. It is the sin above others of the Church of Rome, that it has assumed to itself that name of Priest, which none in heaven or in earth is worthy to bear but the Son of God, and that its ministers pretend to stand between the creature and the Creator in the exercise of His priestly office among men.

\hypertarget{the-sacrifice-of-the-mass-and-other-forms-of-the-sacrificial-theory}{%
\section{The Sacrifice Of The Mass, And Other Forms Of The Sacrificial Theory}\label{the-sacrifice-of-the-mass-and-other-forms-of-the-sacrificial-theory}}

The claim to the possession of a real priesthood, and to the power of making and presenting to God a real propitiatory sacrifice, is fundamental to the theory of the Church of Rome, and is one of the great pillars on which its spiritual strength leans. The right to stand between God and man in the character of mediator, to exercise the priest's office in place of Christ on the earth, to negotiate as man's intercessor with God, and to arrange the terms of his acceptance or condemnation, to make and offer the sacrifice which alone can avail unto justification of life, to retain or remit sin, to give or withhold saving grace,---in short, the claim to the sacerdotal office lies at the very foundation of the Popish system. This one principle of a priestly power existing in her ministry, accompanying all their administrations, and sanctifying all their acts, runs through the whole details of the Church system of Rome, and is the grand secret of very much of its success. We see it fully and conspicuously developed in connection with the Romish doctrine of the Supper, and as the foundation of the sacrifice of the mass. But it is not confined to that one department of the Popish Church system. The sacerdotal principle pervades it, more or less, throughout its entire range; and the Church of Rome has thus added to its many sins the one emphatic sin of usurping the place of Him who has an unchangeable priesthood in heaven and on earth, and of seizing out of His hands the powers that He wields as ``Priest for ever.'' But great and awful though the sin be of arrogating the place and prerogatives of the one High Priest of His people, it is yet a sin which pays its price to the Church that commits it, in the spiritual prestige that it confers, and the spiritual authority that it brings along with it. A sense of the need of some mediator between the sinner and an offended God, a feeling of the absolute necessity of a priest and intercessor for a fallen creature, to negotiate the terms of his pardon and acceptance, can hardly ever be rooted out from the guilty conscience. And the Church of Rome, when it ventures to arrogate to itself on earth that very office which guilty nature needs, and succeeds in its perilous claim to be regarded as the only priest and intercessor between sinners and God, establishes for itself a spiritual dominion over the souls of its victims, greater and more absolute than any other dominion in this world. And hence the tenacity with which the Romish Church clings to the claim of a priestly or sacerdotal office, inseparably connected as it is with some of the most monstrous and incredible pretensions, with the dogma of transubstantiation, with the claim to forgive sin, which none but God can do, with the pretence of making and presenting a Divine and propitiatory sacrifice to the Almighty.

In spite of the explicit abrogation of the office with the abrogation of the Old Testament dispensation; in spite of the palpable inconsistency of the office with the spirit of the Gospel, and the privileges of believers; and, worse still, in spite of the inconsistency of the office with the sole priesthood of Christ, the Church of Rome ordains each one of her ministers to be a priest, and invests him with the power and authority of an earthly priesthood. It needs must be that a priest have a sacrifice to present unto God. ``This man must of necessity have somewhat to offer.'' And having ordained, as she alleges, a real priest, the Church of Rome proceeds to put into his hands a real sacrifice, and gives him warrant to offer it to God for the sins of the living and the dead.

The doctrine of the Church of Rome on this vital point is laid down in such a manner in her authorized formularies that it is impossible to explain it away. The Council of Trent has defined it in such terms, that the attempts made by more modern Romanists to soften down the atrocious dogma of the real offering-up of the sacrifice of the Lord, body and blood, soul and Divinity, in the Sacrament by the priest, are in vain. Speaking of ``the institution of the most holy sacrifice of the mass,'' the Council declares that it is ``a visible sacrifice, as the nature of man requires, by which that bloody one, once to be accomplished on the Cross, might be represented, and the memory of it remain even unto the end of the world.'' And with this statement, expressive of the representative or commemorative character of the ordinance, the apologists of the Church of Rome, whose desire is to conceal the real doctrine held by her on this subject, very often terminate their quotation, as if the Council of Trent held it to be no more than a symbolical sacrifice in memory of Christ's. But that this is not the case, the words of the Council's definition leave us no room to doubt. It proceeds: ``For after the celebration of the old passover, which the multitude of the children of Israel sacrificed in memory of their departure from Egypt, Christ instituted a new passover, even Himself, to be sacrificed by the Church through the priests under visible signs (Seipsum ab Ecclesiâ per sacerdotes sub signis visibilibus immolandum), in memory of His departure out of this world unto the Father, when by the shedding of His blood He redeemed us and snatched us from the power of darkness, and translated us into His kingdom.'' ``And since in this Divine sacrifice, which is performed in the mass, that same Christ is contained and immolated in an unbloody manner, who on the altar of the Cross once offered Himself with blood, the holy Synod teaches that that sacrifice is, and becomes of itself, truly propitiatory; so that if with a true heart and right faith, with fear and reverence, we approach to God, contrite and penitent, we may obtain mercy and find grace to help in time of need. Wherefore the Lord, being appeased by the offering of this, and granting grace and the gift of repentance, remits crimes and sins, even great ones. For it is one and the same victim,---He who then offered Himself on the Cross being the same Person who now offers through the ministry of the priests, the only difference being in the manner of offering (Una enim eademque est hostia, idem nunc offerens sacerdotum ministerio, qui Seipsum tunc in cruce obtulit, sola offerendi ratione diversa).'' And, once more: ``If any shall say that the sacrifice of the mass is only one of praise and thanksgiving, or a bare commemoration of the sacrifice which was made upon the Cross, but not propitiatory; or that it only profits him who receives it, and ought not to be offered for the living and the dead, for sins, pains, satisfactions, and other necessities,---let him be accursed.''

There are two things in regard to the doctrine of the Church of Rome put beyond all dispute or cavil by these statements. First, it is Christ Himself transubstantiated into the elements, and corporeally present in the Sacrament, that is offered up by the priest as a real sacrifice. It is utterly impossible for Romanists to escape from this dogma so long as the language of Trent remains uncancelled. No attempt can succeed to give it a mystical or symbolical meaning, and soften down the authoritative assertion of the Council, that in the Supper there is a real sacrifice of Christ Himself by the priest. Romish controversialists may indeed adopt different modes of explaining how the sacrifice of the mass stands related to the sacrifice of the Cross. Some of them, like Harding the Jesuit, in his reply to Bishop Jewel, may plainly and unhesitatingly assert ``that Christ offered and sacrificed His body and blood twice,---first in that holy Supper, unbloodily, when He took bread in His hands and brake it, and afterwards on the Cross with shedding of His blood.'' Others of them, like Möhler, in his Symbolism, with a view to make the doctrine less palpably inconsistent with Scripture, may assert another form of it, and maintain that there are not two sacrifices, but one, and that the sacrifice of the Supper constitutes a part of that sacrifice which Christ offered on the Cross; or, to use Möhler's own language, ``Christ's ministry and sufferings, as well as His perpetual condescension to our infirmity in the Eucharist, constitute one great sacrificial act, one mighty action undertaken out of love for us, and expiatory of our sins, consisting, indeed, of various individual parts, yet so that none by itself is, strictly speaking, the sacrifice.'' ``The will of Christ to manifest His gracious condescension to us in the Eucharist, forms no less an integral part of His great work than all besides, and in a way so necessary, indeed, that whilst we here find the whole scheme of redemption reflected, without it the other parts would not have sufficed for our complete atonement.'' But however Romanists may choose to explain it,---whether as a repetition of the sacrifice of the Cross, or a continuation of it,---the Supper is unquestionably, according to the doctrine of the Church of Rome, a real sacrifice, made up of Christ's body and blood. And second, this real sacrifice is truly propitiatory in its nature, having virtue in it to satisfy Divine justice, and to constitute a proper atonement for sin. These two doctrinal positions are clearly and undeniably laid down by the Council of Trent, and in such a manner that Romanists cannot evade them. And it is certainly one cause of thankfulness, and no small one, that the Council of Trent was overruled by Divine Providence to put this and other of the monstrous tenets of Romanism into such a dogmatic and articulate form, that it is now utterly impossible for the Church of Rome to deny or escape from them.

What, then, are we to say to the real sacrifice asserted by the Church of Rome, a true propitiation to God for sin, repeated day after day by countless priests who have authority and power to make and offer it?

\hypertarget{the-doctrine-of-the-church-of-rome-is-in-direct-contradiction-to-the-doctrine-of-scripture-which-declares-that-there-is-one-priest-and-no-more-than-one-under-the-gospel.}{%
\subsection{The doctrine of the Church of Rome is in direct contradiction to the doctrine of Scripture, which declares that there is one Priest, and no more than one under the Gospel.}\label{the-doctrine-of-the-church-of-rome-is-in-direct-contradiction-to-the-doctrine-of-scripture-which-declares-that-there-is-one-priest-and-no-more-than-one-under-the-gospel.}}

``Sacrifice and priesthood,'' say the Fathers of the Council of Trent, ``are so joined together by the ordinance of God, that they existed under every dispensation.'' There can be no doubt that the statement is correct in this sense, that wherever there is a sacrifice, there must be a priest to offer it, and wherever there is a priest, he must of necessity have a sacrifice to offer.2 And hence, as part of the sacrificial theory of the Supper and essential to it, the ordination by which the Church of Rome sets apart persons for the work of the ministry includes, as its main and characteristic feature, a commission not to preach the Gospel and to dispense its ordinances, but to make and offer sacrifices to God for the souls of men. Hers is mainly and distinctively an order of priests, and not an order of ministers,---a succession from age to age of sacrificers and intercessors, and not of preachers. And thus her system is distinctively opposed to the system of Scripture, which points to one Priest, and forbids our lips to name a second in the Gospel Church. The argument of the last section might be sufficient, without further illustration, to establish this. But the point is so vital, and it is brought out with such power and effect by the Apostle Paul, that I cannot help adverting to his statements on this subject.

The grand design of that magnificent exposition of the doctrine of Christ's office and nature and work in the Epistle to the Hebrews, is to prove that, far above and beyond the mediators and priests under the law, Christ was the one Son and the one Priest of God, in a way and manner altogether exclusive and peculiar, and such as to contrast Him with all others who ever, in any secondary sense, bore these names. In regard to the priesthood more especially, there were under former dispensations two orders of priests, with one of which the apostle compares our Lord, with the other of which the apostle contrasts Him; and both the comparison and the contrast serve to bring out more distinctly the singular and exclusive character that He bears as the Priest of God, who has neither partner nor successor in the office. There was, according to the apostle, a priesthood after the order of Melchisedec, and there was a priesthood after the order of Aaron. With the priesthood after the order of Melchisedec our Lord is compared. There was room in that order for but one Priest, and no more than one; and for this reason, as stated by the apostle, ``He abideth a Priest continually.'' In the office that he held He had no predecessor, and He had no successor. Melchisedec stood alone in the typical order that bears his name; and the more surely and distinctly to mark out this singularity of his position, we are told, with respect to his office, that he was ``fatherless, motherless, ungenealogied, having neither beginning of days nor end of life'' (ἀπατωρ, ἀμητωρ, ἀγενεαλογητος, μητε ἀρχην ἡμερων μητε ζωης τελος ἐχων). And such as the type was, so is the Antitype. The Lord Jesus Christ was ``made a Priest after the order of Melchisedec;'' and, like that of His type, His office is singular and exclusive; He knows neither predecessor nor successor in it; having not only in His Divine nature, but in His mediatorial character, ``neither beginning of days nor end of life.'' None went before, and none shall come after this Priest; or, as the apostle expresses it, His office is one ``that passeth not from Him to any other.''2 The comparison instituted between our Lord's priesthood and that of Melchisedec demonstrates that He is the one Priest, with none to go before or succeed Him in that character.

But again, with the priesthood of Aaron that of our Lord is contrasted by the apostle; and the contrast serves to bring out in like manner the very same grand doctrine. In that priesthood there were not one, but many priests, following each other in rapid succession. The mortal and dying men who inherited the blood and the office of Aaron ``were not,'' as the apostle tells us, ``suffered to continue by reason of death.'' One after another passed away in swift succession, so that in the not lengthened period of the Aaronic Church there were truly ``many priests,'' following each other rapidly in office, as ever and anon death removed them from beside the altar where they sacrificed and interceded. With them our Lord is contrasted, and not compared in this respect. ``This man, because He continueth ever, hath an unchangeable priesthood.'' ``He is consecrated for evermore.'' He is endued with ``the power of an endless life,'' and ``ever liveth to make intercession for His people.'' Compared with the order of Melchisedec, and contrasted with the order of Aaron, our Lord is emphatically marked out as the one Priest of God, who can have none to follow, even as He had none to go before Him in His office. And the many priests, anointed day by day continually, and succeeding each other in rapid succession in the Church of Rome, are most decisively declared to be inconsistent with His one glorious priesthood.

\hypertarget{the-popish-theory-of-the-lords-supper-is-in-direct-opposition-to-the-doctrine-of-scripture-which-declares-that-there-is-one-sacrifice-and-no-more-than-one-under-the-gospel.}{%
\subsection{The Popish theory of the Lord's Supper is in direct opposition to the doctrine of Scripture, which declares that there is one sacrifice, and no more than one, under the Gospel.}\label{the-popish-theory-of-the-lords-supper-is-in-direct-opposition-to-the-doctrine-of-scripture-which-declares-that-there-is-one-sacrifice-and-no-more-than-one-under-the-gospel.}}

This argument is likewise brought out with commanding force and effect---as if by way of anticipation of the very error of the Papacy---in Paul's Epistle to the Hebrews. He exhibits the contrast between the many priests under the law and the one Priest of God under the Gospel, immortal, and living ever to discharge that office of priesthood in which He had no predecessor and can have no follower, and in which, like Melchisedec, He stood alone. But in close relation with this, he exhibits the contrast also between the many sacrifices under the law with their ceaseless repetition, and the one sacrifice of the Lord Jesus Christ, which never was, and never could be, repeated. The argument by which the apostle demonstrates the unspeakable superiority of the sacrifice of Christ over the sacrifices offered by the sons of Aaron, is a brief and decisive one. The very fact of the repetition of the one, and the non-repetition of the other, was the conclusive evidence of that superiority. The sacrifices under the law were repeated day by day continually; the priest had never done with offering, and the altar never ceased to be wet with the blood of the victims. What was done to-day had to be repeated to-morrow; and the sacrifice was never so completely made and finished but that it had to be repeated afresh, and renewed times without number. And why? The reason was obvious. They were essentially imperfect. They could never so accomplish the great object of atoning for sin but that their renewal was necessary; and what was done on one day had to be supplemented by what was to be done on the next. ``The law,'' says the apostle, ``having a shadow of good things to come, and not the very image of the things, can never with those sacrifices which they offered year by year continually make the comers thereunto perfect. For then would they not have ceased to be offered? because that the worshippers once purged should have had no more conscience of sins. But in those sacrifices there is a remembrance again made of sins every year.'' The fact of their ceaseless repetition was the evidence of their essential imperfection. But in contrast with this, and as an evidence of its sufficiency, the apostle urges the consideration that the sacrifice made by Christ was offered up once, and no more than once. It stood alone, as an offering made once for all, and never again to be repeated,---a sacrifice so complete in its single presentation that it admits of no repetition or renewal. Christ cannot die a second time upon the Cross, as if His first death were incomplete in its efficacy or its merits; for ``by one offering He has perfected for ever them that are sanctified'' or atoned for. Again and again the apostle renews his argument, and his assertion of the fact on which the argument is founded. ``Christ was once offered to bear the sins of many.'' ``Nor yet that He should offer Himself often as the high priest.'' ``For then must He often have suffered since the foundation of the world.'' ``He entered in once into the holy place;'' and ``we are sanctified through the offering of the body of Jesus Christ once for all.'' ``By one offering He hath perfected for ever them that are sanctified.'' The argument is decisive. The perfection of Christ's sacrifice, and the non-repetition of Christ's sacrifice, are inseparable. If that sacrifice needs to be repeated, then it cannot be perfect.

And the reasoning of the apostle is conclusive, as if by anticipation, against the many sacrifices of the Church of Rome in the Supper, whatever explanation may be adopted by its advocates to explain away the contradiction between their practice and the doctrine of Scripture. Let the sacrifice of the mass be a repetition of the sacrifice of Christ upon the Cross, as some Romanist controversialists hold it to be,---and their explanation plainly and undeniably means, that the sacrifice of the Cross needs to be repeated day by day, in order to accomplish the salvation of sinners. Or, let the sacrifice of the mass be a continuation of the sacrifice of Christ on the Cross, and a part of the same atonement, as other Romanists expound it,---and this explanation plainly and undeniably means, that the sacrifice of the Cross was not finished when Christ bowed His head and gave up the ghost. Explain the connection as you will between the sacrifice of the mass and the atonement made upon the Cross, it is utterly inconsistent with the argument of the apostle by which he proves the unapproachable perfection of Christ's work, from its being that one offering which never can be repeated or followed by another.

\hypertarget{what-is-essential-to-the-very-nature-of-a-true-propitiatory-sacrifice-is-awanting-in-the-pretended-sacrifice-of-the-mass.}{%
\subsection{What is essential to the very nature of a true propitiatory sacrifice is awanting in the pretended sacrifice of the mass.}\label{what-is-essential-to-the-very-nature-of-a-true-propitiatory-sacrifice-is-awanting-in-the-pretended-sacrifice-of-the-mass.}}

What was offered on the altar in former times could be no propitiatory sacrifice to God unless it was dedicated to Him by death. Believing sacrifice itself to be a positive institution of God, we must look for the nature and import of the observance only in His Word, and in the practice sanctioned by His appointment. And taking the case of the Old Testament sacrifices, we are warranted in saying that they were uniformly dedicated to God by death, and that ``without shedding of blood there could be no remission.'' There were, indeed, offerings under the law not connected with the shedding of blood, and not accompanied by the destruction of life; but these were not propitiatory. In every case of a propitiatory offering the victim was slain, and the atonement made through the shedding of blood. Expiation and the death of the offering---atonement and shedding of blood---were so inseparably connected, that there could be no real sacrifice of a propitiatory nature when the sacrifice was not dedicated to God by death. From the very earliest times blood was accounted a holy thing, not to be eaten or made use of for common purposes; and the very terms of the prohibition explain the reason of it: ``For the life of the flesh is in the blood, and I have given it to you upon the altar to make an atonement for your soul; for it is the blood that maketh atonement for the soul.''3 Without blood shed there could be no expiation. And here lies one difficulty of the Romish dogma of the sacrifice of the mass. It is a propitiation for the sins of the living and the dead; it is no bare commemoration of a sacrifice, but itself a sacrifice, with virtue to satisfy Divine justice and atone for sin; it is an offering of expiation offered wherever there is a priest to consecrate the ordinance and present it to God. It is a sacrifice of Christ, offered up in propitiation of His Father's righteous displeasure, and efficacious for the remission of sin. But yet we are assured by the apostle that ``Christ dieth no more; death hath no more dominion over Him. For in that He died, He died unto sin once: but in that He liveth, He liveth unto God.'' The Lord Jesus Christ, in His glorified human nature, has long since passed away from the scene of His suffering and humiliation; seated at the Father's right hand, He has rested Him from His work of sorrow and blood, and can repeat no more the agony of the Garden or of the Cross. He does bear with Him indeed in heaven, impressed for ever on His human flesh, the tokens of suffering and crucifixion; ``as a lamb that has been slain,'' He appears on high in the sight of His Father and His angels, marked with the visible evidence of sacrifice and death. But He repeats the sacrifice no more; His blood is not afresh poured out. The proofs of His once finished sacrifice which He carries about in His person are enough; and with these silent but eloquent witnesses to make good His cause, He pleads the virtue of that sacrifice, and never pleads in vain. His uninterrupted and continual advocacy, founded on the merits of His one sacrifice, all-sufficient and complete, supersedes the necessity of its repetition; He needs to die no more for the many sins of His people, which they daily renew, because He once died a death enough for them all, and now lives a life of everlasting intercession, based upon that death, for His people. Without shedding of blood, without atoning suffering, without life rendered as expiation for life, the pretended sacrifice of the mass is inconsistent with the scriptural idea of sacrifice dedicated to God by death.2

Upon such grounds as these we are warranted to say that the sacrificial theory of the Church of Rome, more fully developed in her dogma of the mass, but running throughout her whole spiritual system, is entirely opposed to the doctrine of the Word of God, which asserts, as fundamental to the Gospel, that as there is but one Priest, so there is but one sacrifice known in the New Testament Church. But there are various modifications of this sacrificial theory which, avoiding the extreme doctrine of the Papacy, are held by many semi-Romanists, and still assert that the Lord's Supper is a sacrifice. There are two of these held very commonly by High Churchmen in the English Establishment, to which I would very briefly advert.

1st, In a sense very different from the Romish, it was held by not a few of the Christian Fathers in the early centuries,---and the doctrine has been revived in more recent times in the Church of England,---that the elements of bread and wine were a true material sacrifice, not indeed propitiatory, but eucharistic; very much in the same way as the first fruits laid upon the altar by appointment of the Mosaic law, were a thank-offering to God for the overflowing of His bounties to His creatures. According to this view, the elements of bread and wine, offered to God in the Supper as a material sacrifice without blood, are the fulfilment of the prophecy of Malachi, in which he foretells, in regard to Gospel times, that ``a pure offering,'' as contradistinguished from the bloody sacrifice of the law, should then be offered in to God's name. ``From the rising of the sun to the going down of the same, Thy name shall be great among the Gentiles; and in every place incense shall be offered unto Thy name, and a pure offering.'' This sacrificial theory of the Supper is certainly free from the vital and most fundamental error of the Church of Rome, when it ascribes to the sacrifice in the ordinance a propitiatory character; but it is open to insurmountable objections.

First, a material sacrifice, in the sense of a thank-offering to God for the bounties of His providence, has not the slightest countenance in any of those passages of the New Testament which describe the nature and design of the Supper. It is hardly anything else than a conceit, gratuitously invented by those who saw that it was impossible to regard the Supper as a propitiation for sin, but who were anxious, in conformity with the unguarded language of the patristic writers on the subject, to devise some plausible excuse for applying the term ``sacrifice'' to the Supper. Second, the theory is entirely inconsistent with the first and primary characteristic of the Supper, as clearly laid down in Scripture, namely, that it is an ordinance commemorative of the propitiatory sacrifice of Christ. Third, the theory of a material sacrifice in the Supper, in the sense of a thank-offering of bread and wine for the bounties of Providence, is repugnant to the spiritual nature of the Gospel dispensation, which stands opposed to typical worship.

2d, There is another sacrificial theory of the Supper, much more common than the one now mentioned, and indeed, with various but unimportant modifications, the prevalent theory among those High Churchmen of the English Establishment who reject the extreme views of Popery, as asserted in the doctrine of the mass, but who hold that in the Supper there is a real propitiatory sacrifice, and a real sacrificing priest. According to this view, the elements of bread and wine, not transubstantiated, but remaining unchanged, become, by the words of institution and the consecration of the priest, the body and blood of Christ symbolically and mystically; in consequence of the sacramental union between the sign and the thing signified in the Sacrament, the elements are both to God and to us equivalent to and of the same value with Christ Himself; and the offering up to God of the elements, thus both representing a crucified Saviour, and not inferior in virtue or worth to the Saviour Himself, becomes a true propitiatory sacrifice made to the Almighty for sin. Upon this theory of the Supper, the office of priest in the Christian Church is similar to that of priest under the law: both offer to God real, although symbolical sacrifices, equally pointing to Christ,---there being this difference, that the Aaronic priesthood offered a sacrifice of blood in the prospect of the Saviour's sacrifice to come; while the Christian priesthood offers an unbloody sacrifice in memory of the Saviour's sacrifice now past; and also, that the sacrifices presented now in the Supper, in consequence of their sacramental union with Christ, are infinitely more precious than the sacrifices of the former economy. Such, briefly, and so far as I am able to understand it, is the prevalent doctrine among the majority of the High Church party in the Church of England at the present day, who are not yet prepared, as an extreme section of them appear to be, to accept the Tridentine definitions of the nature and efficacy of the Sacrament of the Lord's Supper. It is maintained and expounded at length in a work recently republished in the Anglo-Catholic Library, entitled, The Unbloody Sacrifice and Altar Unveiled and Supported, by Johnson.

This theory, while excluding the dogma of transubstantiation, which Romanists feel to be necessary to give consistency and foundation to their doctrine of the Supper, approaches in other essential respects very closely to that doctrine, asserting, as it does, a real sacrificing priest and a real propitiatory sacrifice in the Supper. The principles already laid down in opposition to the Popish theory of the Supper are almost all equally available against the now mentioned modification of it. It is subversive of the whole doctrine and character of the Gospel. Under the Christian dispensation there is no priest but One, and He is in heaven. It is His incommunicable name, which none in heaven or on earth may bear but Himself. There is no sacrifice or propitiation but one, and that was finished on the Cross erected upon Calvary, looking back, as it does, for thousands of years over the long array of bloody offerings, which were but the types that pointed towards it, not yet come; and looking forward, as it does, over the long array of ordinances in the Christian Church, commemorative of it, now that it is past. Neither type beforehand, nor commemoration afterhand, could share in its character as an expiatory sacrifice for sin. There is now no dedication of victims to God by death,---life given for life, and blood exchanged for blood,---in order to make a propitiation. The tragedy of the Cross cannot now be renewed, nor atoning blood be shed afresh; and yet ``without the shedding of blood there is no remission'' in Sacrament or in sacrifice. Under whatever form, or modification the sacramental theory be held, which asserts in the Supper a real sacrifice, and a true propitiation for sin, it is a dishonour done to the Lamb of God, who ``by the one offering of Himself has perfected for ever them that are sanctified,'' and who, in virtue of that one Divine offering, now ``liveth for ever to make intercession for His people.''

\bibliography{book.bib,packages.bib}

\end{document}
